\section{Blokų grandinės technologija}

Blokų grandinė - tai vieno su kitu susijusių blokų grandinė, kurios blokuose saugomi
nekeičiami įrašai \cite{SatoshiNakamoto}. Šią technologiją galima apibūdinti kaip daugybę paskirstytų nekintamų skaitmeninių įrašų 
(angl. \textit{immutable distributed ledger}), tarpusavyje susietų taikant kriptografiją (blokų grandinės pavyzdys pateikiamas~\ref{fig:blockchain} paveiksle). Technologija geriausiai žinoma dėl jos panaudojimo Bitcoin kriptovaliutoje.
Šiame skyriuje apžvelgiami pagrindiniai blokų grandinės techniniai aspektai, savybės bei galimi skirtingi variantai.

\begin{figure}[H]
    \centering
    \includegraphics[scale=0.6]{img/blockchain}
    \caption{Supaprastintas blokų grandinės modelis}
    \label{fig:blockchain}
\end{figure}

\subsection{Nekintamumas}

Blokų grandinėje kiekvienas blokas yra sudarytas iš šių dalių:

\begin{itemize}
    \item transakcijų. Kiekviena transakcija yra duomenys, kuriuos norima saugoti blokų grandinėje. Šie duomenys gali būti bet kokia vertinga informacija:
    finansinės transakcijos, programinis kodas, asmens duoti sutikimai (angl. \textit{consents}) ar kt. Kiekviena transakcija yra pasirašoma
    kūrėjo privačiu raktu. Vienas blokas gali turėti vieną arba daugiau transakcijų;
    \item bloko kriptografinės maišos funkcijos rezultato (angl. \textit{hash});
    \item ankstesnio (tėvinio) bloko kriptografinės maišos funkcijos rezultato;
    \item bloko sukūrimo laiko. Blokai grandinėje saugomi chronologiškai;
    \item kitų metaduomenų (pvz. bloko eilės numerio, blokų grandinės versijos, \textit{nonce} darbo įrodymui).
\end{itemize}

Kiekvieno bloko maišos funkcijos rezultatas priklauso nuo jo transakcijų, prieš tai buvusio bloko maišos rezultato ir bloko metaduomenų.
Jeigu betkurio bloko duomenys būtų pakeisti, tuomet maišos funkcija sugeneruotų kitokį maišos rezultatą ir būtų lengva patikrinti, kad
naujai perskaičiuotas maišos rezultatas nesutampa su bloke esančiu rezultatu. Taip pat, kadangi kiekvienas blokas
priklauso nuo prieš tai buvusio bloko, net ir pakeitus vieną iš pirmųjų blokų, pakeitimas būtų pastebimas pridedant naujus blokus ir būtų galima suprasti,
kad turima blokų grandinės versija yra nevalidi (žr.~\ref{fig:blockchainNotIntact} pav.). Tokiu būdu kiekvienas blokų grandinės blokas patvirtina prieš tai
buvusio bloko integralumą, taip pasiekiant blokų grandinės nekintamumą (angl. \textit{immutability}),
nes perrašyti įrašus blokuose nepastebėtam labai sunku \cite{SatoshiNakamoto}.

\begin{figure}[H]
    \centering
    \includegraphics[scale=0.6]{img/blokchainNotIntact}
    \caption{Bloko grandinėje validavimas}
    \label{fig:blockchainNotIntact}
\end{figure}

\subsection{Decentralizuotumas}

Blokų grandinės sistema yra decentralizuota - nėra vieno centrinio serverio, kuris vienas turėtų visą blokų grandinę.
Sistemą sudaro daugybė blokų grandinės mazgų (angl. \textit{node}), kurie turi visą blokų grandinės kopiją. Šie mazgai
yra atsakingi už naujų transakcijų validavimą, blokų su transakcijomis kūrimą, sukurtų blokų priėmimą į blokų grandinę ir pranešimus kitiems mazgams apie naują į grandinę priimtą
bloką \cite{Antonopoulos2016}. Kiekvienas mazgas yra susietas su keletu kitu mazgų. Mazgas, kuris nori pridėti naują bloką (vadinamas \textit{kasėju}), praneša apie jį
kitiems mazgams, jie savo ruožtu žinią perduoda kitiems mazgams ir taip ilgainiui kiekvienas mazgas tinkle turi naujausią blokų grandinės versiją.

Kadangi nėra centrinės institucijos, kuri nuspręstų, ar siūlomas blokas yra tinkamas priimti į grandinę, sprendimą bendrai turi priimti
visi tinklo dalyviai. Egzistuoja skirtingos taisyklės, vadinamos konsensuso strategijomis (plačiau apie juos~\ref{blockchain:consensus} skyrelyje), kuriomis
remdamiesi tinklo mazgai nusprendžia, ar pasiūlytas blokas yra validus. Šios taisyklės apibrėžia, kaip tinklo dalyviai turi įrodyti bloko validumą jį siūlydami į grandinę
bei kaip patikrinti kito dalyvio pasiūlyto bloko validumą.

\subsection{Skirtingi tipai}

Priklausomai nuo tinklo dalyviams suteikiamų blokų grandinės skaitymo ir rašymo teisių,
išskiriami trys pagrindiniai blokų grandinės tipai: vieša, konsorciumo bei privati.  Tipų skirtumai pateikiami~\ref{tab:blockchainComparison} lentelėje.

% \caption{Viešos, konsorciumo ir privačios blokų grandinės palyginimas \cite{Zheng2017}}
% \label{tab:blockchainComparison}
% Table generated by Excel2LaTeX from sheet 'BlockchainComparison'
\begin{table}[htbp]
    \centering
    \caption{Viešos, konsorciumo ir privačios blokų grandinės palyginimas \cite{Zheng2017}}
      \begin{tabular}{|p{8.11em}|l|p{10.22em}|p{6.055em}|}
      \hline
      \multicolumn{1}{|r|}{} & \textbf{Vieša} & \textbf{Konsorciumo} & \textbf{Privati} \bigstrut\\
      \hline
      \textbf{Konsensuso nustatymas} & Visi kasėjai & Išrinkti tinklo dalyviai & Viena organizacija \bigstrut\\
      \hline
      \textbf{Skaitymo teisės} & Viešos & Gali būti viešos ar apribotos & Gali būti viešos ar apribotos \bigstrut\\
      \hline
      \textbf{Centralizuotumas} & Nėra & Dalinis & Yra \bigstrut\\
      \hline
      \textbf{Efektyvumas} & Mažas & Didelis & Didelis \bigstrut\\
      \hline
      \end{tabular}%
    \label{tab:blockchainComparison}%
  \end{table}%
  

Kadangi vieša blokų grandinė yra atvira visam pasauliui, visiems matomos ir joje išsaugotos transakcijos. Tai sudaro puikias salygas įrašų
auditui, tačiau sumažina naudotojų privatumą. Siekiant išlaikyti tam tikrą privatumo lygį, viešoje grandinėje matomi tik transakcijas atlikusių 
asmenų vieši raktai \cite{SatoshiNakamoto}.

Privačios bei konsorciumo blokų grandinės yra tik dalinai decentralizuotos - jose blokų validavimą ir priėmimą į grandinę atlieka vienas
ar dalis tinklo dalyvių. Šios grandinės privalumai: visi validuotojai yra žinomi, grandinės efektyvesnės dėl greičiau priimamų blokų, apribotos blokų skaitymo teisės suteikia didesnį
privatumo lygį, o iškilus poreikiui, tinklo dalyviai gali pakeisti ar atšaukti įvykusias transakcijas \cite{Buterin2015}. Konsorciumo ir privačios blokų grandinės labiau tinkamos įmonių vidiniam (ar jungtiniam,
pvz. tarp kelių finansinių institucijų) naudojimui. Blokų grandinių karkasų sprendimus įmonėms siūlo IBM, Microsoft, Hyperledger \cite{Zheng2017}.

\subsection{Konsensuso strategijos} \label{blockchain:consensus}

Kadangi blokų grandinės sistema yra decentralizuota, nėra centrinės institucijos, kuri nuspręstų, ar naujai siūlomas pridėti į grandinę blokas
yra validus (be transakcijų su falsifikuotais duomenimis). Todėl blokų grandinės tinkle taikoma konsensuso strategija,
pagal kurią nusprendžiama, ar pridėti naują bloką į grandinę. Apžvelgiamos trys dažnai naudojamos strategijos: darbo, įtakos bei autoriteto įrodymo.

\subsubsection{Darbo įrodymo (angl. \textit{proof of work})}

Darbo įrodymo konsensuso strategija remiasi principu, kad daug pastangų ir resursų į bloko validumo įrodymą įdėjęs tinklo
dalyvis nebus linkęs sukčiauti. Šioje strategijoje tinklo dalyvis, norėdamas pridėti bloką į blokų grandinę, turi išspręsti laikui ir resursams
imlų matematinį
uždavinį (užsiima \textit{bloko kasimu}). Pirmas uždavinio reikšmę radęs tinklo dalyvis praneša apie ją kitiems, kurie turi patvirtinti,
ar ši reikšmė teisinga. Jei tai patvirtinta, tinklo dalyviai patikrina, ar naujojo bloko transakcijos yra validžios. Jeigu jos validžios,
blokas pridedamas į grandinę \cite{Zheng2017}.

Darbo įrodymo matematinis uždavinys dažniausiai būna paremtas kriptografine maišos funkcija, kurios rezultatą lengvą validuoti,
tačiau duomenis, sugeneravusius šį rezultatą, sunku surasti. Uždavinio tikslas - surasti šiuos duomenis. Tinklo dalyviai eikvoja didžiulius
kiekius elektros energijos ir laiko,
nes radimas būna paremtas duomenų perrinkimu (angl. \textit{brute force}). Dėl šios priežasties rezultato ieškantiems \textit{kasėjams}
 neretai būna įvesta paskatinimo sistema, kuri teisingą reikšmę radusį
\textit{kasėją} apdovanoja piniginiu atlygiu \cite{SatoshiNakamoto}. 

Kadangi blokų grandinės tinklas yra decentralizuotas, įmanoma situacija, kad labai panašiu metu į grandinę skirtingų mazgų pridėti du validūs blokai.
Taip dalis tinklo dalyvių gaus vieną mazgą, o dalis - kitą, o abu jie bus susieti su tuo pačiu prieš tai buvusiu bloku.
Tokiu atveju, taikoma ilgiausios grandinės taisyklė (žr.~\ref{fig:blockchainLongestRule} pav.). Mazgai dirba prie pirmiau gauto bloko,
tačiau išsaugo kitą gautą bloką kaip šaką. Po to, kai bus gautas dar vienas blokas, jis bus susietas tik su viena iš šakų - taip ši šaka taps ilgesnė. Tuomet
ilgesnė šaka paskelbiama aktyviąja grandine, visi su trumpesniąja šaka dirbę mazgai turi pereiti prie aktyviosios grandinės, o atmesto bloko (vadinamo \textit{bloku-našlaičiu}) transakcijos
grąžinamos į bendrą transakcijų sankaupą (angl. \textit{transaction pool}) \cite{SatoshiNakamoto}.
Realiuose blokų grandinės taikymuose, dažnai laukiama keleto iš eilės einančių naujų blokų, kad būtų galima atmesti \textit{bloką-našlaitį}. Pavyzdžiui,
Bitcoin blokų grandinėje laukiama apytiksliai 6 blokų, kad \textit{bloką-našlaitį} būtų galima atmesti \cite{Zheng2017}.

\begin{figure}[H]
    \centering
    \includegraphics[scale=0.6]{img/blockchainLongestRule}
    \caption{Ilgiausios grandinės taisyklės taikymas}
    \label{fig:blockchainLongestRule}
\end{figure}

Pagrindinis šios konsensuso strategijos privalumas yra tas, kad dideli kasimo kaštai gali atgrasyti programišius nuo potencialių atakų. Tačiau,
taip veltui išeikvojama daugybė elektros energijos - skaičiuojama, kad kasimas Bitcoin ir Ethereum blokų grandinėms kartu sudėjus per dieną
sueikvoja elektros energijos, kurios vertė yra apie 1 milijoną dolerių \cite{Ethereum}. Taip pat, dėl ilgo uždavinio sprendimo laiko vieno bloko priėmimas į grandinę
gali užtrukti - Bitcoin blokų grandinėje tai užima apie 10 minučių \cite{Zheng2017}.

\subsubsection{Turto įrodymo (angl. \textit{proof of stake})}

Turto įrodymo konsensuso strategija remiasi principu, kad daug blokų grandinės turto turintis kasėjas
bus sąžiningas, nes išaiškinus jo nesąžiningumą jis rizikuoja prarasti savo turimą turtą \cite{Baars2016}. Šis algoritmas patiki sprendimą priimti
tiems tinklo dalyviams, kurie įrodo kad turi daugiausia turto (pvz. blokų grandinės kriptovaliutos). Tai gali pasirodyti kaip nesąžiningas sprendimas,
nes turtingiausias tinklo dalyvis gali būti vienvaldžiu sprendimų priėmėju. Dėl to blokų grandinės tinklai neretai taiko šios strategijos
variantus: Peercoin papildomai vertina turto amžių, Blackcoin kitą patvirtintoją paskiria pagal atsitiktinę funkciją, kuri atsižvelgia ir į turimą turtą \cite{Zheng2017}.

Ši strategija leidžia nebeeikvoti didžiulių elektros kiekių, skirtingai nei darbo įrodymo strategija \cite{Ethereum}. Algoritmo efektyvumas
taip pat sutaupo laiko ir blokai būna greičiau patvirtinami ir pridedami į grandinę. Tačiau, dėl praktiškai nulinių bloko \textit{kasimo} sąnaudų,
galimos dažnesnės tinklo atakos \cite{Zheng2017}. 

\subsubsection{Autoriteto įrodymo (angl. \textit{proof of authority})}

Autoriteto įrodymo strategija remiasi keletu tinklo dalyvių, kuriems suteikta teisė validuoti naujus blokus. Ši strategija nebėra tinkama visiškai
decentralizuotai blokų grandinei, kurioje būtinas pilnas pasitikėjimo padalinimas \cite{ProofOfAuthority}. Tačiau ši strategija tinkama privačioms
ar konsorciumo blokų grandinėms.

Šis konsensuso mechanizmas remiasi iš anksto išrinktais tinklo dalyviais, kurie bus atsakingi už blokų validavimą. Kiekvieną kartą pridedant naują
bloką į grandinę, vienas iš išrinktų validuotojų patvirtins arba atmes pasiūlytą bloką. Siekiant sumažinti galimą kenksmingų patvirtintojų žalą,
įvedamos taisyklės, neleidžiančios tam pačiam validuotojui patvirtinti keleto blokų iš eilės \cite{ProofOfAuthority}.

Autoriteto įrodymo strategijoje validuotojams svarbu išlaikyti gerą reputaciją - susigadinus ją, validuotojas gali būti pašalintas iš tinklo. Šis konsensuso mechanizmas
leidžia greitai ir su itin mažais ištekliais pridėti blokus, tačiau nėra tinkamas pilnai decentralizuotoms blokų grandinėms. Šią strategiją
taiko Parity blokų grandinė \cite{ProofOfAuthority}.

\subsection{Pavojai ir trūkumai}

Blokų grandinės technologija sukuria sąlygas decentralizuotai, nesuteikiant pasitikėjimo vienam ar keliems dalyviams, laikyti nekeičiamus duomenis.
Tai atveria įvairių galimybių finansų, daiktų interneto, reputacijos sistemų, saugumo bei privatumo srityse \cite{Zheng2017}, tačiau ši technologija turi ir trūkumų.
Pagrindiniai iš jų aptariami šiame skyriuje.

\subsubsection{Daugumos ataka}

Viešose blokų grandinėse dauguma (>50\%) mazgų tinkle turi patvirtinti bloką, kad jis būtų priimtas į grandinę. Potencialus įsilaužėlis gali pateikti
falsifikuotą blokų grandinės bloką (pvz. su netikromis transakcijomis), tačiau kol jis neturi daugumos skaičiavimo galios tinkle, šis blokas bus
atmestas likusių dalyvių mazge (ir taps \textit{bloku-našlaičiu} taikant ilgiausios grandinės taisyklę). Tačiau jeigu įsilaužėlis (ar keletas jų)
turi daugumą skaičiavimo galios, jis gali dirbti su falsifikuota blokų grandinės versija greičiau negu likę dalyviai tinkle ir taip ilgiausia grandine, prie kurios
pereis visi dalyviai, taps jo sukurta grandinė su falsifikuotais blokai \cite{Zheng2017}. Ši ataka neištiko dviejų didžiausių blokų grandinių Bitcoin bei Ethereum, tačiau
buvo įvykdyta prieš Verge blokų grandinę \cite{Sedgwick2018}.

\subsubsection{Plečiamumas}

Blokų grandinės plečiamumas matuojamas pagal du kriterijus: transakcijų pralaidumą ir saugojimo reikalavimus mazgams. Transakcijų pralaidumas, kuris priklauso nuo to,
kaip greitai nauji blokai su transakcijomi yra pridedami į blokų grandinę, susijęs su taikoma konsensuso strategija.
Kuo taikoma strategija leidžia greičiau priimti naują bloką, tuo greičiau transakcijos bus patvirtintos. Bitcoin blokų grandinėje, taikančioje darbo įrodymo konsensusą,
apdorojoma apie 7 transakcijas per sekundę \cite{Zheng2017}, kai Tendermint blokų grandinė, taikanti
atsparumo klaidoms konsensusą, teigia galinti apdoroti tūkstančius transakcijų per sekundę \cite{Tendermint2017}. Dėl darbo įrodymo strategijos neefektyvumo
blokų grandinės dažnai keičia ją į kitą konsensuso strategiją - Ethereum blokų grandinė ketina pereiti prie turto įrodymu grįsto konsensuso \cite{Ethereum}.

Kita priežastis, dėl ko transakcijos tvirtinamos gana lėtai - blokai turi dydžio apribojimus. Dėl šių apribojimų, tik dalis susikaupusių transakcijų gali būti priimtos
į naują bloką, o likusios turi laukti, kol pateks į kitą bloką. Tam spręsti pasitelktos šalutinės blokų grandinės. Jos dalį bloko duomenų iškelia į šalutinę grandinę,
taip palikdamos daugiau vietos pagrindinės grandinės bloke. Tokį sprendimą priėmė Bitcoin blokų grandinė, į tinklą pristatydama SegWit protokolą, kuris iškelia skaitmeninių
parašų duomenis į atskirą grandinę \cite{Segwit}.

Visos blokų grandinės dydis taip pat gali sukelti plečiamumo problemų. Šiuo metu Bitcoin blokų grandinė užima per 100 gigabaitų \cite{Zheng2017}. Siekiant sumažinti šį kiekvienam mazgui 
reikiamą saugoti duomenų kiekį,
siūlomi įvairūs sprendimai: mazgams neturėti seniausių blokų grandinės dalių (taip seni blokai su transakcijomis būtų iškeliami į atskirą duombazę)
arba dalį blokuose esančios informacijos saugoti šalutinėje blokų grandinėje. 

