\section{Blokų grandinės technologija}

Blokų grandinė - tai vieno su kitu susijusių blokų grandinė, kurios blokuose saugomi
nekeičiami įrašai \cite{SatoshiNakamoto}. Šią technologiją galima apibūdinti kaip daugybę paskirstytų nekintamų skaitmeninių įrašų 
(angl. \textit{immutable distributed ledger}), tarpusavyje susietų taikant kriptografiją. Įrašuose gali būti saugoma bet kokia vertinga informacija:
finansinės transakcijos, programinis kodas, asmens duoti sutikimai (angl. \textit{consents}) ar kt. Technologija geriausiai žinoma dėl jos panaudojimo Bitcoin kriptovaliutoje.
Šiame skyriuje apžvelgiamos pagrindiniai blokų grandinės techniniai aspektai, savybės bei galimi skirtingi variantai.

\subsection{Nekintamumas}

čia įdėt diagramą kaip šiaip atrodo blockchain, hashai ir t.t.

\subsection{Decentralizacijos galimybės}

\subsection{Viešumo tipai}

\subsection{Konsensuso mechanizmas}