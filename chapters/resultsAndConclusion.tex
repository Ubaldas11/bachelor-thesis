\sectionnonum{Rezultatai ir išvados}

\subsection*{Rezultatai}

\begin{enumerate}
    \item Įvardyti naudotojų poreikiai skaitmeninės tapatybės valdymui internete ir apžvelgti
    pagrindiniai naudojami identiteto valdymo modeliai.
    \item Apibendrintos esminės naudotojams kylančios problemos tapatybės valdyme: nepakankama asmens duomenų
    kontrolė, mažėjantis pasitikėjimas paslaugų gebėjimu saugiai valdyti asmens duomenis, didelė priklausomybė nuo tapatybės tiekėjų.
    \item Pristatyta blokų grandinės technologija ir ją naudojantys skaitmeninės tapatybės projektai.
    Nustatyta, kad galimybė
    decentralizuotai saugoti asmens duomenis bei aprašyti duomenų suteikimo taisykles viešuose, nekintančiuose
    ir veikimą atskleidžiančiuose išmaniuosiusiuose
    kontraktuose suteikia pagrindo
    naudoti šią technologiją skaitmeninės tapatybės valdymo srityje. 
    \item Remiantis į naudotoją orientuotos tapatybės principais,
    parengti reikalavimai skaitmeninės tapatybės atributų valdymo modeliui, akcentuojantys įvardytas naudotojų problemas.
    \item Pagal iškeltus reikalavimus paruoštas blokų grandine paremtas tapatybės atributų valdymo modelis, suteikiantis asmeniui galimybę kontroliuoti savo asmens duomenis
    bei autorizuoti paslaugų prieigą prie jų.
    \item Sukurtas pristatyto modelio prototipas. \enquote{Solidity} programavimo kalba realizuoti \enquote{Ethereum} blokų grandinei pritaikyti
    tapatybės saugyklos bei paslaugų registro kontraktai. Paruošti atributų valdymo mobiliosios programėlės maketai.
    \item Įvertintas pristatytas atributų valdymo modelis. Įvardyti jo privalumai, trūkumai,
    galima eksploatavimo kaina bei pritaikymo barjerai.
\end{enumerate}

\subsection*{Išvados}

Blokų grandinės technologija gali būti pritaikyta skaitmeninės tapatybės valdyme. Decentralizuotai saugant asmens duomenis,
panaikinama priklausomybė nuo tapatybės tiekėjo. Viešuose, visiems prieinamuose
išmaniuosiuose kontraktuose aprašius paslaugų autorizavimo logiką, naudotojas gali įsitikinti,
kad jis geba kontroliuoti paslaugų turimas prieigas prie jo asmens duomenų. Dėl blokų grandinės įrašų ir išmaniųjų kontraktų kodo
nekintamumo, asmuo gali būti tikras, kad laikui bėgant sąlygos nepasikeis ir naudotojo priimti
autorizavimo sprendimai išliks.

Aprašytas blokų grandinės taikymas tapatybės valdymo sistemoje turi iššūkių.
Decentralizuotas kodo vykdymas yra gana brangus, įmanomos blokų grandinės tinklo daugumos
atakos gali atgrasyti potencialius naudotojus, o tapatybės duomenų iškėlimas į blokų grandinę verčia
paslaugų tiekėjus kurti papildomą integraciją. Nepaisant to, galimybė grąžinti asmenims jų duomenų kontrolę internete
ir atgauti naudotojų pasitikėjimą gali būti pakankamos paskatos taikyti blokų grandinę skaitmeninės tapatybės valdyme.