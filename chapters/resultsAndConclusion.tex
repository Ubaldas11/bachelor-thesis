\sectionnonum{Rezultatai ir išvados}

Darbe pasiekti \textbf{rezultatai}:

\begin{enumerate}
    \item Apžvelgus internete
    naudojamus skaitmeninės tapatybės valdymo modelius, apibendrintos esminės naudotojams kylančios problemos: nepakankama asmens duomenų
    kontrolė, mažėjantis pasitikėjimas paslaugų gebėjimu saugiai valdyti asmens duomenis, didelė priklausomybė nuo tapatybės tiekėjų.
    \item Pristačius blokų grandinės technologiją ir ją naudojančius skaitmeninės tapatybės projektus, nustatyta, kad decentralizuotas
    asmens duomenų saugojimas bei duomenų suteikimo taisyklių aprašymas išmaniuosiuose kontraktuose
    suteikia pagrindo naudoti blokų grandinę skaitmeninės tapatybės atributų valdymui.
    \item Parengti reikalavimai skaitmeninės tapatybės atributų valdymo modeliui, akcentuojantys įvardytas naudotojų problemas.
    \item Pagal iškeltus reikalavimus paruoštas blokų grandine paremtas tapatybės atributų valdymo modelis, suteikiantis naudotojui galimybę kontroliuoti savo asmens duomenis
    bei autorizuoti paslaugų prieigą prie jų.
    \item Realizuoti pristatyto modelio atributų saugyklos bei paslaugų registro išmanieji kontraktai, parašyti \enquote{Solidity} programavimo kalba \enquote{Ethereum} blokų grandinei.
    Sukurtas atributų valdymo mobiliosios programėlės interaktyvus prototipas.
    \item Įvertintas pristatytas atributų valdymo modelis. Įvardyti jo privalumai, trūkumai,
    galima eksploatavimo kaina bei pritaikymo barjerai.
\end{enumerate}

\hfill \break
Išanalizavus rezultatus gautos darbo \textbf{išvados}:

\renewcommand{\labelenumii}{\arabic{enumii}.}
\begin{enumerate}
    \item Blokų grandinės naudojimas tapatybės atributų valdyme suteikia šiuos pranašumus:
    \begin{enumerate}
        \item Decentralizuotai saugant asmens duomenis,
        panaikinama priklausomybė nuo tapatybės tiekėjo.
        \item Viešuose, visiems prieinamuose
        išmaniuosiuose kontraktuose aprašius paslaugų autorizavimo logiką, naudotojas gali įsitikinti,
        kad jis geba kontroliuoti paslaugų turimas prieigas prie jo asmens duomenų.
        \item Dėl blokų grandinės įrašų ir išmaniųjų kontraktų kodo
        nekintamumo, asmuo gali būti tikras, kad laikui bėgant sąlygos nepasikeis ir naudotojo priimti
        autorizavimo sprendimai išliks.
    \end{enumerate}
    \item Blokų grandinės naudojimas tapatybės atributų valdyme turi šiuos trūkumus:
    \begin{enumerate}
        \item Decentralizuotas kodo vykdymas ir duomenų saugojimas išmaniuosiuose kontraktuose yra gana brangus.
        \item Norint naudotis modeliu, paslaugų tiekėjai yra priversti turėti papildomą integraciją su blokų grandine.
        \item Įmanoma viešos blokų grandinės tinklo daugumos ataka gali atgrasyti potencialius naudotojus.
    \end{enumerate}
    \item Įvertinus nustatytus pranašumus bei trūkumus, galima teigti, kad blokų grandinę skaitmeninės tapatybės atributų valdymui verta naudoti
    tiems naudotojams, kuriems yra itin svarbi jų asmens duomenų kontrolė internete. Plačiau taikyti modelį vertėtų
    sumažinus eksploatavimo kainą bei patobulinus paslaugų atpažinimo procesą.
\end{enumerate}

\hfill \break
Tolimesni galimi darbai:

\begin{enumerate}
    \item Siekiant sumažinti modelio eksploatavimo kainą, ištirti alternatyvas duomenų saugojimui ne blokų grandinėje. Pačioje grandinėje
    paliekant tik nuorodą į duomenų saugojimo vietą, modelio naudojimas taptų pigesnis.
    \item Analizuoti paslaugų identifikavimo alternatyvas. Dabartinis būdas modelyje naudoti paslaugų registrą verčia paslaugų tiekėjus atlikti
    papildomą žingsnį. Taip pat, naudotojas turi pasikliauti atributų valdymo programos atliekamu paslaugos atpažinimu bei registracijos kodo generavimu.
    \item Ištirti paslaugų tiekėjų galimybes naudoti modelį. Supažindinus paslaugų tiekėjus su atributų valdymo modeliu,
    atsižvelgti į jų nuomonę apie galimą naudojimą bei siūlomus patobulinimus.
\end{enumerate}