\sectionnonum{Rezultatai ir išvados}

\textbf{Darbo rezultatai:}

\begin{enumerate}
    \item Apžvelgti pagrindiniai skaitmeninės tapatybės valdymo būdai internete.
    \item Įvardytos esminės naudotojams kylančios problemos tapatybės valdyme: nepakankama asmens duomenų
    kontrolė, mažėjantis pasitikėjimas paslaugų duomenų valdymu ir didelė priklausomybė nuo tapatybės tiekėjų.
    \item Pristatyta blokų grandinės technologija. Nustatyta, kad galimybė kurti
    decentralizuotus, veikimą atskleidžiančius išmaniuosius kontraktus blokų grandinėje suteikia pagrindo
    naudoti blokų grandines ir išmaniuosius kontraktus skaitmeninės tapatybės valdymo srityje. 
    \item Remiantis į naudotoją orientuotos tapatybės (angl. \textit{user-centric identity}) principais,
    parengti reikalavimai skaitmeninės tapatybės atributų valdymo modeliui, akcentuojantys įvardytas naudotojų problemas.
    \item Paruoštas blokų grandine paremtas tapatybės atributų valdymo modelis, įgyvendinantis iškeltus reikalavimus
    ir suteikiantis asmeniui galimybę kontroliuoti savo asmens duomenis
    bei autorizuoti paslaugų prieigą prie jų. Modelis įvertintas išskiriant jo
    privalumus, trūkumus bei galimus pritaikymo barjerus.
    \item Sukurtas pristatyto modelio prototipas. \enquote{Solidity} programavimo kalba realizuoti \enquote{Ethereum} blokų grandinei pritaikyti
    išmanieji kontraktai, įgyvendinantys paslaugų autorizavimo ir atributų saugojimo logiką. Paruošti mobiliosios programėlės, skirtos
    naudotojui palengvinti sąveiką su blokų grandine, maketai.
\end{enumerate}

\textbf{Darbo išvados:}

Skaitmeninės tapatybės valdymas yra sritis, kurioje naudotojai neturi pakankamai kontrolės.
Dideli kiekiai asmens duomenų yra perduodami tarp programų neinformuojant naudotojų
ar nutekinami dėl programišių įsilaužimų. Skaitmeninio identiteto centralizavimas tapatybės tiekėjų sistemose
sukuria situaciją, kurioje naudotojas tampa visiškai priklausomas nuo tapatybės tiekėjo bei yra priverstas
pasitikėti juo.

Blokų grandinės technologija gali padėti išspręsti susidariusias problemas. Decentralizuotai saugant asmens duomenis,
panaikinama priklausomybė nuo tapatybės tiekėjo. Viešuose, visiems prieinamuose
išmaniuosiuose kontraktuose aprašius paslaugų autorizavimo logiką, naudotojas gali įsitikinti,
kad jis geba kontroliuoti paslaugų turimas prieigas prie jo asmens duomenų. Dėl kontrakto kodo
nekintamumo, asmuo gali būti tikras, kad laikui bėgant sąlygos nepasikeis ir naudotojo priimti
autorizavimo sprendimai išliks.

Aprašytas blokų grandinės taikymas tapatybės valdymo sistemoje turi iššūkių.
Decentralizuotas kodo vykdymas yra gana brangus, įmanomos blokų grandinės tinklo daugumos
atakos gali atgrasyti potencialius naudotojus, o tapatybės duomenų iškėlimas į blokų grandinę verčia
paslaugų tiekėjus kurti papildomą integraciją. Nepaisant to, galimybė grąžinti asmenims jų duomenų kontrolę internete
ir atgauti naudotojų pasitikėjimą gali būti pakankamos paskatos taikyti blokų grandinę skaitmeninės tapatybės valdyme.