\sectionnonumnocontent{Santrauka}
Šiame darbe nagrinėtas blokų grandinės technologijos tinkamumas skaitmeninių tapatybių valdymui. Apžvelgus esamų
identifikavimo sprendimų savybes, tirta blokų grandinė ir jos charakteristikos, kurios leistų įveikti dabar kylančius
naudotojų atpažinimo iššūkius.


Nustatyta, kad blokų grandinė gali būti taikoma skaitmeninės tapatybės valdymui \textcolor{red}{tokioje srityje}, kur ši technologija
padeda išspręsti \textcolor{red}{tokias bėdas}. Sukurtas pateikto sprendimo prototipas, parašytas su \textcolor{red}{Language3000}
programavimo kalba.

\raktiniaizodziai{autentifikavimas, tapatybės atpažinimas, skaitmeninė tapatybė, skaitmeninės tapatybės valdymas, blokų grandinė}   

% English version

\sectionnonumnocontent{Summary}
In this paper, blockchain applicability for digital identity management was investigated. After an overview of current
identification solution properties, blockchain and its characteristics were examined, in order to find out whether the technology is suitable
to overcome present user identification challenges.


It was determined that blockchain can be used for digital identity management in \textcolor{red}{this/these fields}, where it allows to
solve \textcolor{red}{those issues}. A prototype of the solution was presented, which was written using \textcolor{red}{Language3000}
programming language.
\keywords{authentication, identity recognition, digital identity, digital identity management, blockchain}
