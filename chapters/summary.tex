\sectionnonumnocontent{Santrauka}

Šiame darbe nagrinėtas blokų grandinės technologijos tinkamumas skaitmeninės tapatybės valdyme. Pristačius naudotojų poreikius identiteto valdymui internete,
apžvelgtas esamų modelių gebėjimas juos išpildyti. Apibendrinus liekančias naudotojų problemas, tirta blokų grandinė ir jos charakteristikos,
kurios leistų įveikti kylančius iššūkius skaitmeninės tapatybės valdyme.

Nustatyta, kad blokų grandinė gali būti taikoma skaitmeninės tapatybės valdymui internete. Pristatytas blokų grandine ir išmaniaisias kontraktais
paremtas atributų valdymo modelis,
leidžiantis naudotojams kontroliuoti savo asmens duomenis ir jų sklaidą. Sukurtas pateikto modelio veikimą pagrindžiantis prototipas.

\raktiniaizodziai{skaitmeninė tapatybė, skaitmeninės tapatybės valdymas, blokų grandinė, išmanieji kontraktai.}   

% English version

\sectionnonumnocontent{Summary}
In this paper, blockchain applicability for digital identity management was investigated. After presenting user needs for identity management,
currently used identity management models were researched. Following that, blockchain and its characteristics were examined,
in order to find out whether the technology is suitable
to overcome present user identification challenges.   a

It was determined that blockchain can be used for digital identity management in the internet. A blockchain and smart contract based attribute management model was presented,
which allows users to take control of their own personal data and it's distribution. A prototype of the model was presented,
which was written in \enquote{Solidity} programming language.

\keywords{digital identity, digital identity management, blockchain, smart contracts.}
