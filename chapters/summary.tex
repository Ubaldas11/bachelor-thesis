\sectionnonumnocontent{Santrauka}
Naudotojo autentifikavimas yra svarbus nemažai daliai programų sistemų. Natūralus siekis
įsitikinti naudotojo autentiškumu kelia vis didesnes problemas - tam pačiam asmeniui besinaudojant
vis didesniu programinės įrangos kiekiu, skaitmeninių tapatybių valdymas sudėtingėja.
Kiekvienas naudotojas tampa priverstas prisiminti aibę prisijungimo
vardų ir slaptažodžių, neretai dėl to pradeda naudoti identiškus slaptažodžius ir paaukoja jų stiprumą, o sistemų kūrėjai turi
skirti papildomų resursų identifikavimo valdymui.


Programų sistemose skaitmeninėms tapatybėms valdyti naudoti įvairūs metodai - atskiros naudotojų duomenų bazės,
centralizuotos platformos kelioms sistemoms, vienkartinio prisijungimo sprendimai. Blokų grandinės yra nauja
alternatyva skaitmeninių identitetų organizavimui programose. Šiame darbe nagrinėtas blokų grandinės
tinkamumas skaitmeninių tapatybių valdymui, \textcolor{red}{pridėti gautus dalykus}.

\raktiniaizodziai{autentifikavimas, tapatybė, skaitmeninis tapatybių valdymas, blokų grandinė}   

% English version

\sectionnonumnocontent{Summary}
User authentication is crucial for a bunchful of software systems. A natural wish to
ensure user's authenticity causes ever bigger problems - as the same person uses more and more software,
digital identity management becomes complex. Every user is forced to remember a lot of different sets of credentials,
that often results in identical passwords and reduced credential strength, when software creators have to put
in additional resources for identity management.

Various methods are used to manage digital identities in software systems - separate user databases,
centralized credential platforms for several systems, single sign-on solutions. Blockchain is a new
alternative to organize identity data in software. This thesis investigates the suitability of blockchain
for digital identity management, \textcolor{red}{add discovered stuff}.
\keywords{authentication, identity, digital identity management, blockchain}