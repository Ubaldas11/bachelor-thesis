\section{Skaitmeninės tapatybės valdymo apžvalga}

\subsection{Tapatybės patvirtinimo poreikis}

Šiais laikais naudotojo identifikavimas yra svarbi interneto taikomųjų
programų dalis. Paslaugų tiekėjai identifikuoja savo naudotojus norėdami \cite{RalucaBudiu2014}:

\begin{itemize}
    \item registruoti (angl. \textit{log}) naudotojų veiklą,
    \item užtikrinti, kad naudotojas iš tikrųjų yra asmuo, kuris sakosi esąs,
    \item suteikti dalį funkcionalumo tik autorizuotiems naudotojams,
    \item individualizuoti tinklalapio ar taikomosios programos turinį pagal naudotojo poreikius,
    \item sukurti paslaugos naudotojų bendruomenę,
    \item išvengti galimų anoniniminių naudotojų atakų.
\end{itemize}

Dėl išvardytų priežasčių naudotojų identifikavimas atlieka svarbią rolę įvairiose taikomųjų programų
srityse - elektroninėje valdžioje, elektroninėje komercijoje, verslo sumanume
(angl. \textit{business intelligence}), tyrimuose bei saugume
(angl. \textit{homeland security}) \cite{Glasser2009}. Kiekvienas paslaugų tiekėjas turi pasirinkti,
kaip autentifikuoti, ir, jei reikia, autorizuoti naudotojus. Programos kūrėjas taip pat turi saugoti
naudotojų suteiktus asmens duomenis ir užtikrinti jų saugumą, o naudotojui tenka rūpintis skirtingų turimų
paskyrų priežiūra ir savo duomenų sklaida tarp skirtingų sistemų. Minimus tapatybės atpažinimo
skaitmeninėje erdvėje aspektus nagrinėja skaitmeninės tapatybės valdymo disciplina.

\subsection{Skaitmeninės tapatybės valdymo samprata}
