\section{Skaitmeninės tapatybės valdymo apžvalga}

\subsection{Tapatybės patvirtinimo poreikis}

Šiais laikais naudotojo identifikavimas yra svarbi interneto taikomųjų
programų dalis. Paslaugų tiekėjai identifikuoja savo naudotojus norėdami \cite{RalucaBudiu2014}:

\begin{itemize}
    \item registruoti (angl. \textit{log}) naudotojų veiklą,
    \item užtikrinti, kad naudotojas iš tikrųjų yra asmuo, kuris sakosi esąs,
    \item suteikti dalį funkcionalumo tik autorizuotiems naudotojams,
    \item individualizuoti tinklalapio ar taikomosios programos turinį pagal naudotojo poreikius,
    \item sukurti paslaugos naudotojų bendruomenę,
    \item išvengti galimų anoniniminių naudotojų atakų.
\end{itemize}

Dėl išvardytų priežasčių naudotojų identifikavimas atlieka svarbią rolę įvairiose taikomųjų programų
srityse - elektroninėje valdžioje, elektroninėje komercijoje, verslo sumanume
(angl. \textit{business intelligence}), tyrimuose bei saugume
(angl. \textit{homeland security}) \cite{Glasser2009}. Kiekvienas paslaugų tiekėjas turi pasirinkti,
kaip autentifikuoti, ir, jei reikia, autorizuoti naudotojus. Programos kūrėjas taip pat turi saugoti
naudotojų suteiktus asmens duomenis ir užtikrinti jų saugumą, o naudotojui tenka rūpintis skirtingų turimų
paskyrų priežiūra ir savo duomenų sklaida tarp skirtingų sistemų. Minimus tapatybės atpažinimo
skaitmeninėje erdvėje aspektus nagrinėja skaitmeninės tapatybės valdymo disciplina.

\subsection{Skaitmeninės tapatybės valdymo samprata}

Dėl nuolat vykstančios interneto ir jame esančių paslaugų plėtros tapatybių
valdymo uždavinys pastaraisiais metais tapo itin svarbus \cite{Glasser2009}. Sprendžiant šį iššųkį,
sukurta skirtingų skaitmeninės tapatybės valdymo sistemų, siekiančių išspręsti naudotojų
tapatybės atpažinimo problemas. Šioms sistemoms įtaką daro
kiti tapatybę nagrinėjantys mokslai (pvz. sociologija), taip pat jos gali atlikti keletą skirtingų funkcijų, susijusių
su naudotojų tapatybe. Žemiau pateikiama diagrama,
kurioje pavaizduotas tapatybės valdymo sistemų kontekstas bei pagrindinės atliekamos užduotys:

\begin{figure}[H]
    \centering
    \includegraphics[scale=0.8]{img/IDMcontextAndUsecases}
    \caption{Skaitmeninių tapatybių valdymo sistemų kontekstas ir užduotys \cite{Glasser2009}}
\end{figure}

Paveiksle matomos disciplinos turi skirtingą poveikį tapatybių valdymo sistemoms. 
Sociologija padeda apibrėžti tapatybę ir jos atitikmenį skaitmeninėje erdvėje, teisės mokslas nusako tapatybės duomenų naudojimo reikalavimus,
o esamos technologijos formuoja sistemos įgyvendinimo niuansus. Verta pastebėti, kad tapatybės valdymo sistema gali atlikti ne visas
diagramoje nurodomas funkcijas, o tik dalį iš jų. Taip pat, 1-ame paveiksle bei visame darbe naudojamos skaitmeninės tapatybės valdymo sąvokos,
tokios kaip \textit{identifikavimas}, \textit{autentifikavimas} ar \textit{autorizavimas} neretai suprantamos skirtingai, o tai sukelia
vieningos terminologijos trūkumą ir dėl jo kylančius neaiškumus \cite{Glasser2009}. Dėl to skyriuje
\enquote{Sąvokų apibrėžimai} pateikiami darbe dažniausiai naudojamų terminų aiškinimai.

Esamų tapatybių valdymo sistemų architektūros bei veikimo principai yra skirtingi -  S. Clauß ir M.Köhntopp savo tyrime pastebi,
kad nėra vieningo standarto identiteto
valdymo sistemoms \cite{Claus2001}. Tolesniuose skyriuose, apžvelgus naudotojų bei paslaugų tiekėjų reikalavimus
tapatybės valdymo sistemoms, pateikiamos skirtingos technologijos
bei sprendimai, naudojami identiteto valdymui internete, jų privalumai bei trūkumai.

\subsection{Skaitmeninės tapatybės valdymo sistemų charakteristikos}

Skaitmeninės tapatybės valdymas yra plati sritis, kurią galima analizuoti iš skirtingų pusių.
Šiame darbe į skaitmeninių tapatybių valdymą žvelgta iš dviejų perspektyvų: naudotojo bei paslaugų tiekėjo \textcolor{red}{palikt tik naudotojo?}. Abi pusės,
priklausomai nuo savo poreikių, turi jiems aktualias identiteto valdymo sistemų savybes. Apžvelgiant skirtingas tapatybės valdymo sistemas,
didžiausias dėmėsys kreiptas į naudotojams bei paslaugų tiekėjams svarbiausius sistemų bruožus.

Tiek naudotojams, tiek paslaugų tiekėjams itin svarbus yra pasirinkto identiteto valdymo sprendimo patikimumas, t.y., asmens duomenų saugumo užtikrinimas.
Naudotojai nenori savo asmens duomenų nutekėjimo (angl. \textit{data leakage}) internete, tuo tarpu paslaugų tiekėjai negali rizikuoti prarasti naudotojų pasitikėjimo
jų paslauga pasirinkus nesaugų tapatybės valdymo sprendimą. Be patikimumo, naudotojams bei paslaugų tiekėjams aktualios ir 
kitos skaitmeninio identiteto sistemų savybės. Žemiau pateikiami pagrindiniai abiejų pusių poreikiai.
\\

\chapter{\textbf{Naudotojams svarbūs tapatybės valdymo sistemų bruožai:}}

\begin{itemize}
    \item identifikatorių bei slaptažodžių kiekis. Naudotojui vidutiniškai turint 25 paskyras, reikalaujančias slaptažodžių \cite{Florencio2007} bei naudojant
    nuo 2 iki 12-os el. paštų \cite{Gross2007}, jis tampa priverstas prisiminti vis daugiau identifikatorių. Atsimintinų autentifikavimo duomenų kiekiui
    augant, naudotojai yra linkę aukoti saugumą dėl patogumo ir naudoti panašius
    slaptažodžius skirtingose sistemose \cite{Pashalidis2003, Samar1999};
    \item asmens duomenų kontrolė. Pasak Nyderlanduose atliktų tyrimų, naudotojai jaučia, kad nekontroliuoja savo asmens duomenų internete \cite{Baars2016}. Dėl to
    naudotojai pradeda nepasitikėti taikomųjų programų kūrėjais, nes jie pilnai nežino, kokia informacija apie juos kaupiama ir kokioms
    sistemoms ji perduodama;
    \item patogumas (angl. \textit{usability}). Naudotojams skaitmeninės tapatybės valdymas dažnai yra tik pašalinis mechanizmas, reikalingas
    norint pasiekti paslaugą \cite{Dhamija2008}. Dėl šios priežasties sistemos naudojimosi patogumas yra svarbus - kuo tapatybės valdymas yra labiau integruotas
    su asmens jau naudojamomis sistemomis ir kuo mažiau jis reikalauja papildomo naudotojo įsitraukimo, tuo labiau naudotojas bus linkęs pasirinkti šį identiteto valdymo sprendimą.
\end{itemize}

\chapter{\textbf{Paslaugų tiekėjams aktualios tapatybės valdymo sistemų savybės:}}

\begin{itemize}
    \item kaštai, skirti tapatybės valdymui. Priklausomai nuo pasirinkto sprendimo,
    paslaugų tiekėjui gali tekti skirti daug arba mažai resursų (programuotojų, laiko, investavimo į technologijas)
    naudotojų tapatybės valdymo veiksmams užtikrinti;
    \item naudotojo patirties kontrolė (angl. \textit{user experience control}). Paslaugų tiekėjai siekia užtikrinti teigiamą
    naudotojų patirtį dėl geresnio naudojamumo, privatumo ir saugumo \cite{Dhamija2008}, o tapatybės valdymo sistemos veikimo primesti sprendimai (pvz. nukreipimai į kitą tinklalapį) gali daryti
    tam įtaką.
\end{itemize}

\subsection{Esamos skaitmeninių tapatybių valdymo sistemos}

Skirtingos tapatybių valdymo sistemos yra pagrįstos skirtingomis architektūromis, kurios įgalina
konkrečios sistemos veikimą. Šiame skyriuje apžvelgiami 3-ys dažniausiai naudojami tapatybių valdymo sistemų
modeliai bei pateikiami jų įgyvendinimo pavyzdžiai.

\subsubsection{Izoliuotas tapatybių valdymas}

\subsubsubsection*{Modelis}

Izoliuotame modelyje paslaugų tiekėjas yra ir tapatybės tiekėjas, nes visos su tapatybės valdymu
susijusios operacijos yra atliekamos vieno serverio. Tapatybės duomenų saugojimas, autentifikavimas
ir autorizavimas yra įgyvendinti paties paslaugų tiekėjo \cite{Cao2010} Kiekvienas naudotojas turi atskirus identifikatorius
kiekvienam paslaugų tiekėjui. Modelis grafiškai pavaizduotas žemiau esančiame paveiksle:

\begin{figure}[H]
    \centering
    \includegraphics[scale=0.65]{img/IsolatedModel}
    \caption{Izoliuotas skaitmeninės tapatybės valdymas \cite{Cao2010}}
\end{figure}

Pagal izoliuotą modelį, naudotojas turi savo paskyrą kiekvieno jam aktualaus paslaugų tiekėjo sistemoje. Kiekvieną kartą autentifikuojat ar autorizuojant
naudotoją, tai atlieka pats paslaugų tiekėjas, bendraudamas tiesiogiai su naudotoju (jo naršykle). Naudotojui prisijungus prie vieno tinklalapio ir gavus
prieigos raktą, jis gali toliau naudotis tinklalapiu, tačiau prireikus pasinaudoti kitu paslaugų tiekėju, tapatybės atpažinimo veiksmai (autentifikavimas, autorizavimas)
turės būti atlikti iš naujo su šiuo paslaugų tiekėju.

Šis modelis yra gana paprastas paslaugų tiekėjams, tačiau greitai tampa
nebekontroliuojamu naudotojams \cite{Josang2005}. Jis verčia naudotojus turėti daug paskyrų, identifikatorių bei slaptažodžių.
Tie patys tapatybės atpažinimo veiksmai (pvz. autentifikavimas) atliekamas iš naujo su kiekviena reikalinga paslauga. Naudotojui
sunku kontroliuoti savo asmens duomenis, nes jis juos patiki visiems paslaugų tiekėjams, kuriais naudojasi - tai apsunkina 
priežiūrą, kur kokie duomenys perduoti bei asmens duomenų atnaujinimą, kuris turi būti atliekamas su kiekvienu paslaugos tiekėju atskirai.

\textcolor{red}{jei reik, pastraipa paslaugų tiekėjo bruožams}.

\subsubsubsection*{Realizacijos}

Kadangi šis naudotojų autentifikavimo bei autorizavimo modelis naudojamas seniausiai,
yra gana nemažai jį įgyvendinusių taikomųjų programų. Lietuvoje šį modelį naudoja
\enquote{Tiketa}, \enquote{Bilietai.lt}, \enquote{Pigu.lt},
\enquote{Varle.lt}, pasaulyje - \enquote{Booking.com}, \enquote{Skycop}, \enquote{AirBnB} bei kitos platformos. Verta
pastebėti, kad dalis iš jų jau remiasi ne tik savo tapatybės valdymo, bet jau turi į savo sistemas integravę ir papildomų autentifikavimo būdų
(pvz. prisijungimą per \enquote{Facebook} ar \enquote{Google}).

Realizacijų technologiniai sprendimai dažniausiai nėra viešai prieinami.
Šiame modelyje kiekvienas paslaugų tiekėjas yra ir tapatybės tiekėjas, tad nereikia
specialaus protokolo, kuriuo duomenimis apsikeistų šios dvi esybės - visa tai pats nusprendžia
ir įgyvendina paslaugų tiekėjas.

\subsubsection{Centralizuotas tapatybių valdymas}

\subsubsubsection*{Modelis}
Centralizuotame skaitmeninių tapatybių valdyme egzistuoja vienas naudotojo identifikatorius,
naudojamas visų paslaugų tiekėjų \cite{Josang2005}. Paslaugų tiekėjo ir tapatybės tiekėjo funkcijos tampa atskirtos - 
tapatybės tiekėjas rūpinasi naudotojo identiteto valdymu, o paslaugų tiekėjas koncentruojasi į savo paslaugos tobulinimą.
Naudotojui šiame modelyje užtenka vieno identifikatoriaus ir slaptažodžio, su kuriais jis gali prisijungti prie visų to paties
paslaugų tiekėjo paslaugų. Šis modelis iliustruotas ~\ref{fig:isolatedModel}-iame paveiksle.

Iš naudotojo perspektyvos, šis modelis yra patogesnis nei izoliuotas. Naudotojui pakanka turėti vieną identifikatorių,
kuris bus tinkamas visoms paslaugoms. Priklausomai nuo realizacijos, naudotojui gali tekti iš naujo prisijungti prie kiekvienos
paslaugos arba vieną kartą prisijungti prie betkurios iš paslaugų ir tapti autentifikuotu visose kitose paslaugose \cite{Josang2005}. Naudotojo asmens
duomenų kontrolė yra geresnė nei izoliuotame skaitmeninės tapatybės valdyme, nes naudotojas asmens duomenis patiki mažesniam kiekiui esybių
(vietoj kiekvienos paslaugos, kiekvienam paslaugų tiekėjui).

Pagrindinis šio modelio trūkumas yra tas, kad jis galimas naudoti tik vieno paslaugų tiekėjo srityje (angl. \textit{scope}). Tai palieka laisvės paslaugų tiekėjui tapatybės valdymo sistemos
realizavimui (tai gali būti centralizuota platforma įmonės ribose), tačiau naudotojas negali turimo identifikatoriaus
naudoti kito paslaugų tiekėjo sistemose.

\begin{figure}[H]
    \centering
    \includegraphics[scale=0.8]{img/centralizedModel}
    \caption{Centralizuotas skaitmeninės tapatybės valdymas \cite{Cao2010}}
    \label{fig:isolatedModel}
\end{figure}

\subsubsubsection*{Realizacijos}

Centralizuotas modelis tinkamas naudoti vieno paslaugų tiekėjo ribose \cite{Josang2005}. Vienas iš pavyzdžių - 
\enquote{Atlassian} įmonės paslaugos. Naudotojui pakanka turėti vieną \enquote{Atlassian} paskyrą ir jis gali naudotis
skirtingais šio paslaugų tiekėjo produktais, tokiais kaip \enquote{Jira}, \enquote{Confluence}, \enquote{Bitbucket} bei kitais. \enquote{Atlassian}
turi ir vienkartinio prisijungimo sąlygas - paslaugas galima sukonfigūruoti taip, kad prisijungus prie vienos iš jų,
naudotojas automatiškai tampa prisijungęs ir prie kitų paslaugų.

Realizacijų naudojamos technologijos nėra viešai skelbiamos. Kadangi centralizuotame modelyje paslaugų tiekėjas
pats įgyvendina tapatybės tiekėjo modulį, jis pats apsprendžia bendravimą tarp tapatybės ir paslaugų tiekėjo. Priklausomai
nuo skirtingų paslaugų dydžio ir reikmių, šis bendravimas su tapatybės tiekėju gali svyruoti nuo centralizuotos duomenų bazės
naudotojų duomenims iki
paskirstytame tapatybės valdyme naudojamų technologijų (SAML, OpenID).

\subsubsection{Jungtinis tapatybių valdymas}

\subsubsubsection*{Modelis}

Centralizuotas modelis reikalauja visų naudotojų būti tame pačiame domene, tačiau naudotojams reikia
paslaugų ir iš kitų domenų \cite{Cao2010}. Jungtinis (angl. \textit{federated}) tapatybių valdymas yra aibė technologijų
ir procesų, kurie leidžia sistemoms dalintis tapatybės informacija ir deleguoti tapatybės valdymo užduotis
tarp skirtingų paslaugų tiekėjų \cite{Maler2008}. Šis tapatybių valdymo modelis įgalina naudotojus turėti vieną
identifikatorių, kurį gali naudoti skirtingų paslaugų tiekėjų tinklalapiuose. Žemiau pateikiama schema, nusakanti
šio modelio architektūrą:

\begin{figure}[H]
    \centering
    \includegraphics[scale=0.75]{img/federatedModel}
    \caption{Jungtinis skaitmeninės tapatybės valdymas \cite{Cao2010}}
    \label{fig:federatedModel}
\end{figure}

Jungtiniame tapatybės valdyme tapatybės tiekėjas yra atskira sistema, su kuria turi integruotis paslaugų tiekėjas. Tapatybės
duomenys bei su tapatybe susiję veiksmai (autentifikavimas, autorizavimas) yra deleguojami šiai sistemai. Naudotojas turi vieną identifikatorių,
su kuriuo prisijungia tiesiogiai tapatybės tiekėjo puslapyje. Prisijungus šioje sistemoje, naudotojas tampa
autentifikuotas visose paslaugose, kurios palaiko šį tapatybės tiekėją \cite{Maler2008}.

Šis tapatybių valdymo modelis yra gana patogus naudotojams. Jiems užtenka turėti vieną identifikatorių, su kuriuo
gali prisijungti prie tinklalapių iš skirtingų paslaugų tiekėjų. Asmens duomenų saugumas priklauso nuo
bendravimo tarp paslaugų bei tapatybės tiekėjų, o šiame modelyje jis sudėtingesnis nei izoliuotame ar centralizuotame
tapatybių valdyme, nes jungtinis valdymas paremtas tarpdomeniniu bendravimu \cite{Maler2008}. Naudotojo duomenų kontrolė jungtiniame 
modelyje turi tiek privalumų, tiek trūkumų. Viena vertus, naudotojas savo asmens duomenis suteiks mažesniam kiekiui sistemų (tik tapatybės tiekėjams,
vietoj visų paslaugų tiekėjų). Tačiau taip paslaugų tiekėjas tampa vieninteliu nesekmės tašku (angl. \textit{single point of failure}) - programišiams įsilaužus
į tapatybės tiekėjo sistemą, visų paslaugų paskyros tampa prieinamos \cite{Pashalidis2003}. Taip pat, naudotojui gali būti sunku kontroliuoti
savo duomenų sklaidą tarp tapatybės tiekėjo ir skirtingų paslaugų, ką rodo ir \textit{Cambridge Analytica} incidentas \cite{CambridgeAnalytica}.

\subsubsubsection*{Realizacijos}

\subsection{}