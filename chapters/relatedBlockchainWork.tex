\section{Kiti blokų grandinės projektai tapatybės valdymo srityje}

Tiriant blokų grandinės panaudojimą skaitmeninės tapatybės valdymo srityje, rasta įvairių
technologiją bandančių pritaikyti projektų. Šis skyrius skirtas trumpai apžvelgti autoriaus nuomone
įdomiausius projektus, susijusius su tapatybe bei blokų grandinės technologija.

\tocless\subsection{Blockstack}

Blockstack projektas yra kompiuterių tinklas, kolektyviai saugantis globalų
domenų vardų bei jų viešų raktų registrą. Su šiuo registru, Blockstack įgyvendiną
decentralizuotą domenų vardų sistemą, kur kiekvienas norintis gali užregistruoti savo domeną \cite{BlockstackWhitepaper}.
\enquote{OneName} projektas, naudojantis \enquote{Blockstack} sistemą, siekia sukurti decentralizuotą tapatybę,
kurią susikūręs naudotojas galėtų atsisakyti visų kitų turimų socialinių tinklų (\enquote{Facebook},
\enquote{Google}) tapatybių.

\tocless\subsection{UPort}

\enquote{Consensys} įmonės \enquote{UPort} projektas panašus į \enquote{OneName} - jis skirtas naudotojams turėti decentralizuotas
skaitmenines tapatybes, registruotas \enquote{Ethereum} tinkle, su kuriomis galėtų atlikti \enquote{Ethereum} transakcijas.
Naudotojai su \enquote{UPort} paskyra
galėtų prisijungti prie paslaugų bei valdyti savo atributus. Šio darbo rašymo metu \enquote{UPort} turi
parengę integracijas paslaugoms ir elektroninę piniginę, skirtą \enquote{Ethereum} raktų
ir transakcijų valdymui, tačiau mobilioji programėlė UPort skaitmeninei tapatybei valdyti vis dar ruošiama (angl.
\textit{under development}) \cite{UPort}.

\tocless\subsection{Tradle}

\enquote{Tradle} yra blokų grandine paremtas projektas, skirtas finansinėms institucijoms (bankams,
draudimo įmonėms) įgyvendinti \enquote{pažink savo klientą} (angl. \textit{Know Your Customer - KYC}) procesus,
suteikiant klientams didesnę asmens duomenų kontrolę. \enquote{Tradle} suteikia
klientams programinę įrangą, kuri padės integruoti \enquote{Tradle} platformą su jų įmonėse jau egzistuojančia klientų duomenų saugojimo
infrastruktūra. Klientams išreikštinai sutikus, \enquote{Tradle} taip pat suteikia galimybę perduoti asmenų duomenis iš vienos
finansinės įmonės į kitą \cite{Baars2016}.