\section{Tapatybės atributų valdymo modelio vertinimas} \label{section:blockchainIDMevaluation}

Šiame skyriuje vertinamas\hypertarget{section:BCIDM}{~\ref{section:BCIDM}} skyriuje pristatytas blokų grandine paremtas tapatybės atributų
valdymo modelis: kokie jo privalumai, trūkumai, pritaikymo barjerai.

\subsection{Privalumai}

\subsubsection{Naudotojui suteikta atributų valdymo kontrolė}

Naudotojas su atributų valdymo programa gali autorizuoti paslaugas pasiekti jo atributus, esančius blokų grandinėje. Programoje
taip pat galima pamatyti visus įvestus atributus ir paslaugas, kurios yra autorizuotos juos pasiekti. Naudotojas taip pat
gali pakeisti anksčiau priimtą sprendimą ir programa atnaujins prieigą blokų grandinėje. Naudotojui nusprendus pakeisti
atributo reikšmę, jis tai padaro atributų valdymo programoje, o ši kreipiasi į blokų grandinę ir joje atnaujina esančias reikšmes. Kaip
galėtų atrodyti tokia atributų valdymo programa pateikiama priede\hypertarget{appendix:attributeManagementApp}{~\ref{appendix:attributeManagementApp}}.\\
Išvardytos funkcijos padengia\hypertarget{BCIDM:requirements}{~\ref{BCIDM:requirements}} skyrelio 1, 2 ir 3 reikalavimus.

\subsubsection{Decentralizuoti asmens duomenys}

Tapatybės duomenis saugant blokų grandinėje, atributai tampa pasiekiami nepriklausomai nuo vienos trečiosios šalies pasiekiamumo.
Taip tiek paslaugų tiekėjai, tiek naudotojas gali betkada pamatyti suteiktus atributus tol, kol bent vienas blokų grandinės mazgas
yra pasiekiamas. Taip pat, nors duomenys decentralizuoti, dėl jų šifravimo jie išlieka prieinami tik naudotojui ir jo pasirinktoms
paslaugoms.
Šis privalumas įgyvendina\hypertarget{BCIDM:requirements}{~\ref{BCIDM:requirements}} skyrelio 4-ą reikalavimą.

\subsubsection{Sumažintas reikalingas pasitikėjimas trečiosiomis šalimis}

Kadangi tapatybės atributai yra saugomi viešoje, decentralizuotoje blokų grandinėje, kurios išmaniųjų kontraktų
logika prieinama visiems, naudotojams nebereikia pasitikėti vien tik paslaugų bei tapatybės tiekėjais - jų
atributai šiuo atveju priklauso nuo korektiško išmaniųjų kontraktų veikimo.

Verta pastebėti, kad
pristatytame modelyje naudotojui vis dar reikia pasitikėti atributų valdymo programos veikimu. Tačiau, ši programa pati
asmens duomenų nesaugo (tik persiunčia įvestas ir grąžina blokų grandinėje esančias reikšmes). Pasitikėjimui
padidinti ši programa taip pat turėtų būti viešai prieinamo atviro kodo.

Taip pat, paslaugų tiekėjai, parsisiuntę atributą
iš blokų grandinės, vis dar gali jį išsisaugoti. Tačiau, galimybė paslaugai pačiai nesaugoti naudotojo duomenų turėtų būti pagrindinė paskata
naudotis šiuo modeliu - taip naudotojai labiau pasitikės šia paslauga. Galimas išsisaugojimo pavojus pastebimas ir
\cite{MITPaper} tyrime, kurio autoriai aptaria ir galimybę neleisti paslaugai pasiekti pačio duomens, o atlikti
reikalingus skaičiavimus pačiame blokų grandinės tinkle.

\subsection{Trūkumai}

\subsubsection{Paslaugų autentiškumo tikrinimas}

Modelyje aprašytas paslaugų registras padeda apsisaugoti nuo programišių, galinčių apsimesti paslaugomis. Šio registro veikimas
nėra itin patogus, nes kiekviena paslauga turi gauti sugeneruotą kodą iš atributų valdymo programos ir įrašyti jį į atskirą išmanųjį kontraktą.
Tai reikalauja daugiau integracijų (tarp paslaugos, atributų valdymo programos ir papildomo išmaniojo kontrakto)
bei papildomo naudotojo pasitikėjimo atributų valdymo programa: jai patikėta generuoti kodą bei atlikti paslaugos registracijos
tikrinimą.

Galimos kelios alternatyvos paslaugų identifikavimui užtikrinti. Galima būtų naudotis jau egzistuojančiais identiteto
registravimo projektais, kuriuose paslaugos užregistruotų savo tapatybę (pvz. \enquote{NameCoin} ar \hypertarget{section:relatedWork}{\ref{section:relatedWork}}
skyrelyje minimu \enquote{Blockstack}). Kitas būdas, aptariamas \cite{Baars2016} tyrime, būtų turėti registrą, kuriame paslaugų autentiškumą
patvirtintų patikima trečioji šalis (pvz. bankas ar valdžios įstaiga). Tokiu atveju, paslaugos turėtų autentifikuotis šios institucijos
sistemoje, o ji įrašytų į paslaugų registrą apie sėkmingą paslaugos registraciją. Tokiu atveju pasitikėjimas būtų perkeliamas
pasirinktoms institucijoms.

\subsubsection{Nepatikimos atributų reikšmės}

Paslaugų tiekėjai iš blokų grandinės gauna reikšmes, kurias įvedė pats naudotojas. Tai nesudaro jokių keblumų,
kai jų teisingumą gali patikrinti pati paslauga (pvz. atsiųsti žinutę į telefono numerį), tačiau kai to atlikti negalima,
paslaugai tenka pasitikėti naudotojo įvesta reikšme.

Tai būtų galima tobulinti, jei šias reikšmes galėtų verifikuoti
atsakingos institucijos (pvz. bankas). Tuomet, šios institucijos turėtų teikti papildomą paslaugą, kurioje naudotojui
paprašius, jos galėtų atlikti užklausą į blokų grandinę ir patvirtinti, kad atributo reikšmė yra teisinga, o tai būtų
išsaugota išmaniajame kontrakte. Tokiu atveju,
į blokų grandinę besikreipianti paslauga galėtų matyti, ar šis atributas yra verifikuotas. Panašus tvirtinimo mechanizmas
nagrinėjamas \cite{Baars2016} tyrime.

\subsubsection{Kaina}

Blokų grandinės išmaniųjų kontraktų funkcijų kvietimai kainuoja pinigus. Įprastu atveju, funkcijos vykdymą apmoka
funkcijos kvietėjas.
Aprašytame modelyje kvietimus į blokų grandinę vykdo tiek paslaugų tiekėjas, tiek atributų valdymo programa. Kai naudotojas dėl didesnės kontrolės
gali sutikti mokėti už atributų valdymo programos naudojimą, nėra aišku, ar paslaugų tiekėjai būtų pasiryžę mokėti už funkcijų kvietimus.

Vienas iš galimų sprendimų - kviečiant funkciją paslaugai, ją apmokėtų ne paslaugos tiekėjas, o pats kontraktas. Taip
kontraktas būtų apmokamas iš naudotojo lėšų, o jos naudojamos paslaugai kviečiant kontrakto funkcijas.
Tačiau, tokį mechanizmą turi palaikyti pasirinkta blokų grandinė, taikant jį reiktų apsisaugoti ir nuo
galimų pavojingų pakartotinių kvietimų, kurie būtų skirti tik išeikvoti kontrakto lėšas.

\textcolor{red}{kai prototipą pakodinsiu, pažiūrėsiu kiek kainuoja kuro funkcijos, jas gal reiks įmest}

\subsection{Pritaikymo barjerai}

Pagrindiniai sunkumai, kurie kiltų tokį modelį taikant praktikoje, yra galimas paslaugų nenoras bendradarbiauti bei menka blokų grandinės
technologijos branda.

Paslaugų tiekėjams, norint pradėti naudoti šį modelį savo paslaugose, reikia susikurti blokų grandinės paskyrą bei suintegruoti
savo sistemas su blokų grandinės išmaniaisiais kontraktais bei atributų valdymo programa.
Tai gali būti palengvinta sukuriant viešai prieinamą programinę įrangą, paruoštą naudoti paslaugų
tiekėjams, tačiau paslaugų kūrėjai vis tiek turi nuspręsti naudoti šį modelį. Pagrindinė paskata jiems yra didesnis naudotojų pasitikėjimas
suteikiant jiems galimybę kontroliuoti savo duomenis bei pagerintas asmens duomenų pasiekiamumas juos iškėlus į decentralizuotą blokų grandinę.

Blokų grandinė yra vis dar gana nauja technologija. Vieša blokų grandinė gali būti pažeista daugumos atakos, nemažai skeptikų
teigia, kad pagrindinis technologijos panaudojimas išliks tik kriptovaliutoms, technologijos plečiamumas (angl. \textit{scalability}) aktyviai
tiriamas. Taip pat, 2016-aisiais metais išmaniojo kontrakto \enquote{The DAO} trūkumą išnaudojusi ataka privertė \enquote{Ethereum} 
blokų grandinę atlikti išsišakojimą (angl. \textit{hard fork}), kuris pakeitė transakcijų istoriją ir grąžino įsilaužėlio
pavogtą kriptovaliutą jos savininkams \cite{TheDAO}. Tai rodo, kad ypatingomis sąlygomis blokų grandinės nekintamumas gali būti pažeistas\\
Nepaisant išvardytų priežasčių, technologijos populiarumas neblėsta. Blokų grandinės technologiją pradeda naudoti didžiosios kompanijos
(\enquote{Microsoft} bei \enquote{IBM} siūlo
blokų grandinę kaip paslaugą \cite{Zheng2017}). Taip pat, vis daugiau programuotojų kuria išmaniuosius kontraktus
(darbo rašymo metu \enquote{Ethereum} blokų grandinėje yra 25374 verifikuoti išmanieji kontraktai \cite{EthVerifiedContracts}).\\
Blokų grandinė dar nėra brandi, ilgai taikoma technologija. Ji aktyviai bandoma tiek startuolių, tiek
korporacijų, vis dar intensyviai tiriama mokslininkų. Technologiją naudojančių programų sėkmė nulems, ar asmenys
sutiks naudotis blokų grandine paremtomis programomis.




