\section{Tapatybės atributų valdymo modelio vertinimas}

Šiame skyriuje apžvelgiama, kaip~\ref{section:BCIDM} skyriuje pristatytas blokų grandine paremtas tapatybės atributų
valdymo modelis tenkina iškeltus reikalavimus, jo trūkumus ir pritaikymo barjerus.

\subsection{Reikalavimų įgyvendinimas}

Pristatytas modelis įgyvendina~\ref{BCIDM:requirements} skyrelyje aprašytus reikalavimus.
Kadangi asmens duomenys
saugomi decentralizuotuoje blokų grandinėje, paslaugų tiekėjai betkada gali juos pasiekti tol, kol bent vienas blokų grandinės
mazgas yra pasiekiamas.

\subsection{Privalumai}

\subsubsection{Naudotojui suteikta atributų valdymo kontrolė}

Naudotojas su atributų valdymo programa gali autorizuoti paslaugas pasiekti jo atributus, esančius blokų grandinėje. Programoje
taip pat galima pamatyti visus įvestus atributus ir paslaugas, kurios yra autorizuotos juos pasiekti. Naudotojas taip pat
gali pakeisti anksčiau priimtą sprendimą ir programa atnaujins prieigą blokų grandinėje. Naudotojui nusprendus pakeisti
atributo reikšmę, jis tai padaro atributų valdymo programoje, o ši kreipiasi į blokų grandinę ir joje atnaujina esančias reikšmes. Kaip
galėtų atrodyti tokia atributų valdymo programa pateikiama priede~\ref{appendix:attributeManagementApp}.\\
Išvardytos funkcijos padengia~\ref{BCIDM:requirements} skyrelio 1, 2 ir 3 reikalavimus.

\subsubsection{Tapatybės atributai decentralizuoti}

Tapatybės atributus saugant blokų grandinėje, atributai tampa pasiekiami nepriklausomai nuo vienos trečiosios šalies pasiekiamumo.
Taip tiek paslaugų tiekėjai, tiek naudotojas gali betkada pamatyti suteiktus atributus tol, kol bent vienas blokų grandinės mazgas
yra pasiekiamas.
Šis privalumas įgyvendina~\ref{BCIDM:requirements} skyrelio 4-ą reikalavimą.

\subsubsection{Sumažintas reikiamas pasitikėjimas trečiosioms šalims}

Kadangi tapatybės atributai yra saugomi viešoje, decentralizuotoje blokų grandinėje, kurios išmaniųjų kontraktų
logika prieinama visiems, naudotojams nebereikia pasitikėti vien tik paslaugų bei tapatybės tiekėjais - jų
atributai šiuo atveju priklauso nuo korektiško išmaniųjų kontraktų veikimo.

Verta pastebėti, kad pristatytame modelyje naudotojui vis dar reikia pasitikėti atributų valdymo programa. Šiam pasitikėjimui
padidinti ši programa turėtų būti atviro kodo, kuris viešai prienamas. Taip pat, paslaugų tiekėjai parsisiuntę atributą
iš blokų grandinės, vis dar gali jį išsisaugoti. Tačiau, galimybė paslaugai pačiai nesaugoti naudotojo duomenų turėtų būti pagrindinė paskata
naudotis šiuo modeliu - taip naudotojai labiau pasitikės šia paslauga. Galimas išsisaugojimo pavojus pastebimas ir
\cite{MITPaper} tyrime, kurio autoriai aptaria ir galimybę neleisti paslaugai pasiekti pačio duomens, o atlikti
reikalingus skaičiavimus pačiame tinkle.

\subsection{Trūkumai}

\subsubsection{Atributų reikšmių įvedimas}

Norint naudotis aprašytu modeliu, asmuo turi į atributų valdymo programą įvesti norimų atributų reikšmes. Taip daroma todėl,
kad naudotojui nereiktų vesti atributo kiekvieną kartą, kai nori autorizuoti tam tikrą paslaugą. Tačiau, tokiu būdu,
naudotojas suteikia savo duomenis dar vienai paslaugai - pačiai atributų valdymo programai. Tai suteikia patogumo (nereik
daugybę kartų vesti atributo), tačiau naudotojui reikia pasitikėti atributų valdymo programa. Norint to išvengti,
galima būtų naudotojo prašyti įvesti atributą kiekvieną kartą, kai paslauga prašo autorizuoti prieigą prie pasirinkto atributo.

\subsubsection{Nepatikimos atributų reikšmės}

Paslaugų tiekėjai iš blokų grandinės gauna reikšmes, kurias įvedė pats naudotojas. Tai nesudaro jokių keblumų,
kai jų teisingumą gali patikrinti pati paslauga (pvz. atsiųsti žinutę į telefono numerį), tačiau kai to atlikti negalima,
paslaugai tenka pasitikėti naudotojo įvesta reikšme.

Tai būtų galima tobulinti, jei šias reikšmes galėtų verifikuoti
atsakingos institucijos (pvz. bankas). Tuomet, šios institucijos turėtų teikti papildomą paslaugą, kurioje naudotojui
paprašius, jos galėtų atlikti užklausą į blokų grandinę ir patvirtinti, kad atributo reikšmė yra teisinga, o tai būtų
išsaugota išmaniajame kontrakte. Tokiu atveju,
į blokų grandinę besikreipianti paslauga galėtų matyti, ar šis atributas yra verifikuotas. Panašus tvirtinimo mechanizmas
nagrinėjamas \cite{Baars2016} tyrime.

\subsubsection{Kaina}

Blokų grandinės išmaniųjų kontraktų funkcijų kvietimai kainuoja pinigus. Įprastu atveju, funkcijos vykdymą apmoka
funkcijos kvietėjas.
Aprašytame modelyje kvietimus į blokų grandinę vykdo tiek paslaugų tiekėjas, tiek atributų valdymo programa. Kai atributų valdymo
programa gali būti mokama naudotojui, nėra aišku, ar paslaugų tiekėjai būtų pasiryžę mokėti už funkcijų kvietimus.

Vienas iš galimų sprendimų - kviečiant funkciją paslaugai, ją apmokėtų ne paslaugos tiekėjas, o pats kontraktas. Taip
kontraktas būtų apmokamas iš naudotojo lėšų, o jos naudojamos paslaugai kviečiant kontrakto funkcijas.
Tačiau, tokį mechanizmą turi palaikyti pasirinkta blokų grandinė, taikant jį reiktų apsisaugoti ir nuo
galimų pavojingų pakartotinių kvietimų, kurie būtų skirti tik išeikvoti kontrakto lėšas.

\subsection{Pritaikymo barjerai}