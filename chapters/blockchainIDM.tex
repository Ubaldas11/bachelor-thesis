\section{Blokų grandine paremta tapatybės atributų valdymo sistema}

Atsižvelgus į blokų grandinės savybes, tinkamas tapatybės valdymo problemoms spręsti, šiame skyriuje pateikiamas
blokų grandine paremtas skaitmeninės tapatybės valdymo modelis. Modelis skirtas spręsti ~\ref{IDM:problemsSummarized} skyrelyje
apibendrintoms naudotojų problemoms. Kadangi tapatybės valdymas apima
daug skirtingų procesų, aprašomas modelis apsiriboja asmens tapatybės atributų valdymu ir perdavimu.

\subsection{Reikalavimai}

Naudotojams kylančios problemos tapatybės valdyme remiasi į tai, kad jų tapatybė yra
visiškai patikėta valdyti tapatybės tiekėjams. Dėl menko naudotojų įsitraukimo į tapatybės valdymo
procesus, asmenys neretai lieka tik pasyvūs stebėtojai.


Siekiant išspręsti šią problemą, kuriamo modelio reikalavimai kurti pagal į naudotoją orientuotos tapatybės
(angl. \textit{user-centric identity}) principus. Ši paradigma akcentuoja naudotojus kaip centrinę
identiteto valdymo sistemų dalį, perduodant paslaugų ir tapatybės tiekėjų turimą tapatybės kontrolę
naudotojams \cite{Cao2010}. Tokiu būdu, naudotojai turi aktyviau prižiūrėti ir dalyvauti tapatybės
valdymo procesuose (dažniausiai naudojant papildomą programinę įrangą),
tačiau geriau žino, kaip ir kur yra naudojami jų asmens duomenys.

Taikant į asmenis orientuotos skaitmeninės tapatybės principus, reikalavimai modeliui suformuluoti naudotojų istorijų forma:

\begin{enumerate}
    \item Kaip interneto naudotojas, aš noriu žinoti, kurios programos turi prieigą prie kurių mano asmens duomenų.
    \item Kaip interneto naudotojas, aš noriu kontroliuoti savo asmens duomenis ir pats suteikti arba atmesti prieigą paslaugų tiekėjams pasiekti mano duomenis.
    \item Kaip interneto naudotojas, aš noriu galimybės lengvai atnaujinti savo asmens duomenis vienoje vietoje.
    \item Kaip interneto naudotojas, aš nenoriu, jog mano asmens duomenų pasiekiamumas priklausytų tik nuo vienos trečiosios šalies pasiekiamumo.
\end{enumerate}

Pateiktos naudotojų istorijos padengia 1 skyriuje apibrėžtus asmenų poreikius tapatybės valdymo
sistemoms bei išskirtas kontrolės, pasitikėjimo ir skaidrumo trūkumo problemas. Saugumo iššūkiai internete
yra plati tema, kuri šiame skyriuje nebus nagrinėjama. \textcolor{red}{ar užtenka tokio sakinio pasakymui kad čia \textit{out of scope}? nes saugumui užtikrint reik
kad naudojamos paslaugos SSL turėtų, ir šiaip plati tema labai, kurios vien blockchain neišspręs}

\subsection{Architektūra}

Kuriamo modelio architektūra remiasi MIT tyrime \cite{}

\subsection{Naudojimo sekos}