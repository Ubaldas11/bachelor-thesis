\section{Tapatybės atributų valdymo modelis} \label{section:BCIDM}

Atsižvelgus į blokų grandinės savybes, tinkamas tapatybės valdymo problemoms spręsti, šiame skyriuje pateikiamas
vieša blokų grandine paremtas skaitmeninės tapatybės valdymo modelis. Modelis skirtas spręsti\hypertarget{IDM:problemsSummarized}{~\ref{IDM:problemsSummarized}} skyrelyje
apibendrintoms naudotojų problemoms. Kadangi tapatybės valdymas apima
daug skirtingų procesų, aprašomas modelis apsiriboja asmens tapatybės atributų valdymu, saugojimu ir perdavimu.

Modelio architektūra kurta atsižvelgiant į MIT tyrime \cite{MITPaper} pristatytus teorinius protokolus, kurie apibrėžia
mobiliųjų programėlių, naudotojo bei blokų grandinės bendravimą, asmeniui suteikiant telefone saugomą
informaciją (pvz., lokacijos duomenis) naujai parsisiųstai programėlei. Aprašomas modelis modifikuoja bendravimą, siekiant 
jį pritaikyti naudojimui internete, taip pat išplečia funkcionalumą, leidžiant naudotojui pačiam įvesti norimą informaciją.

\subsection{Modeliui keliami reikalavimai} \label{BCIDM:requirements}

Naudotojams kylančios problemos tapatybės valdyme remiasi į tai, kad jų tapatybė yra
visiškai patikėta valdyti tapatybės tiekėjams. Dėl menko naudotojų įsitraukimo į tapatybės valdymo
procesus, asmenys neretai lieka tik pasyvūs stebėtojai.

Siekiant išspręsti šią problemą, kuriamo modelio reikalavimai kurti pagal į naudotoją orientuotos tapatybės
(angl. \textit{user-centric identity}) principus. Ši paradigma akcentuoja naudotojus kaip centrinę
identiteto valdymo sistemų dalį, perduodant paslaugų ir tapatybės tiekėjų turimą tapatybės kontrolę
naudotojams \cite{Cao2010}. Tokiu būdu, naudotojai turi aktyviau prižiūrėti ir dalyvauti tapatybės
valdymo procesuose (dažniausiai naudojant papildomą programinę įrangą),
tačiau geriau žino, kaip ir kur yra naudojami jų asmens duomenys.

Taikant į asmenis orientuotos skaitmeninės tapatybės principus, reikalavimai modeliui suformuluoti naudotojų istorijų forma:

\begin{enumerate}
    \item Kaip interneto naudotojas, aš noriu žinoti, kurios programos turi prieigą prie kurių mano asmens duomenų.
    \item Kaip interneto naudotojas, aš noriu kontroliuoti savo asmens duomenis ir pats pasirinkti, kurios mano naudojamos paslaugos gali pasiekti mano
    asmens duomenis.
    \item Kaip interneto naudotojas, aš noriu galimybės lengvai atnaujinti savo asmens duomenis vienoje vietoje.
    \item Kaip interneto naudotojas, aš nenoriu, jog mano asmens duomenų pasiekiamumas priklausytų tik nuo vienos trečiosios šalies pasiekiamumo.
\end{enumerate}

Pateiktos naudotojų istorijos padengia 1 skyriuje apibrėžtus asmenų poreikius tapatybės valdymo
sistemoms bei išskirtas kontrolės, pasitikėjimo ir skaidrumo trūkumo problemas. Interneto tapatybių valdymo sistemų saugumo atakos (angl.
\textit{cross site scripting}, \textit{phishing}) yra plati tema, verta atskiro tyrimo, tad ji šiame modelyje nebus nagrinėjama.

\subsection{Modelio dalys}

Modelis sudarytas iš trijų dalių: atributų saugyklos išmaniojo kontrakto, paslaugų registro išmaniojo kontrakto
bei atributų valdymo programos.

\subsubsection{Atributų saugyklos išmanusis kontraktas} \label{BCIDM:blockchainFunctions}

Atributų saugyklos išmaniajame kontrakte yra saugomi šifruoti naudotojų atributai
bei jų suteikimo logika. Šis kontraktas atsakingas už:

\begin{itemize}
    \item naudotojo atliekamą prieigų suteikimą. Naudotojas, kviesdamas kontrakto funkciją, gali pasirinkti,
    ar suteikti konkrečiai paslaugai prieigą prie jos norimo atributo. Prieigos yra valdomos ne visos tapatybės, o
    atskirų atributų lygmenyje;

    \item šifruotų atributų saugojimą. Kontrakte saugomi naudotojo tapatybės atributai. Kadangi modelyje naudojama vieša blokų
    grandinė, joje esantys duomenys prieinami visiems - dėl to saugomi šifruoti atributai. Jeigu naudotojas N
    yra autorizavęs paslaugą P pasiekti atributą A, tuomet kontrakte išsaugomas N ir P simetrišku šifro raktu (angl.
    \textit{symmetric encryption key}) užšifruotas atributas A. Taip tik paslauga P
    ir naudotojas N, turintys šifro raktus, galės perskaityti viešame išmaniajame kontrakte esantį atributą;

    \item pateikiamas funkcijas paslaugoms bei naudotojui pasiekti atributus. Išmanusis kontraktas pirmiausia atsižvelgia
    į funkcijos kvietėją. Jei kreipiasi naudotojas, jam grąžinama šifruota atributo reikšmė. Jei kreipiasi paslauga, tikrinama,
    ar naudotojas yra patvirtinęs jos prieigą norimam atributui. Jei taip, grąžinama šifruota reikšmė. Jei ne, grąžinamas pranešimas,
    kad prieiga atmesta. Jeigu naudotojas šios paslaugos prieigos dar nesvarstęs, apie paslaugos P siekį gauti naudotojo N atributą A
    pranešama naudotojui. Kai jis prieigą patvirtins arba atmes, paslauga galės pasiekti atributą A.
\end{itemize}

Atributai kontrakte atskiriami
pagal jų identifikatorius (pvz. 1 - vardas, 2 - telefono numeris ir t.t.).
Siekiant minimizuoti blokų grandinėje laikomą informaciją, atributų identifikatorių
žodynas nesaugomas pačiame kontrakte. Svarbu, kad šis žodynas būtų viešai prieinamas visoms paslaugoms - jis
galėtų būti pateikiamas modelio dokumentacijoje.

\subsubsection{Atributų valdymo programa}

Atributų valdymo programa yra skirta palengvinti naudotojo bendravimą su blokų grandine. Teoriškai, naudotojas galėtų pats formuoti užklausas,
generuoti šifro raktus, perduoti jos paslaugoms, užšifruoti ir dešifruoti siunčiamas reikšmes, tačiau tai nėra patogu. Dėl to ši
programa padeda atlikti \enquote{nematomą darbą} bei supaprastina naudotojo sąsają su bloko grandine. Programoje naudotojas gali:

\begin{itemize}
    \item suteikti arba atmesti paslaugai P prieigą prie atributo A,
    \item peržvelgti suteiktas prieigas ir keisti jas,
    \item įvesti, keisti ar trinti atributų reikšmes.
\end{itemize}

Atributų valdymo programos veikimas panašus į kriptovaliutos piniginės veikimą. Kaip tokia piniginė leidžia naudotojui per vartotojo sąsają
pervesti pinigus iš naudotojo sąskaitos į kitą sąskaitą, taip atributų valdymo programa leidžia naudotojui
valdyti savo paskyros atributus blokų grandinės išmaniajame kontrakte. Programa veikia naudotojo vardu (angl. \textit{on user's behalf}),
turėdama jo privatų raktą
- taip ji gali kreiptis į blokų grandinę
iš naudotojo paskyros, o išmanusis kontraktas, gavęs užklausą iš atributų valdymo programos,
žino, koks naudotojas kreipiasi į funkciją.

Modelyje nėra apibrėžiama, kaip turi būti įgyvendinta ši programa - tai gali būti kompiuterio darbalaukio programa,
mobilioji programėlė ar interneto tinklapis. 

\subsubsection{Paslaugų registro išmanusis kontraktas} \label{BCIDM:serviceRegister}

Šiame modelyje paslaugos yra identifikuojamos pagal jų blokų grandinės paskyros adresą. Atributų valdymo programa,
skaitanti išmaniajame kontrakte laukiančias užklausas, jose mato išsaugotą besikreipusios paslaugos adresą.

Nors adresas yra viešas ir unikaliai identifikuoja kiekvieną blokų grandinės dalyvį, turintį paskyrą, tačiau iš adreso nustatyti tikrąją dalyvio tapatybę yra sudėtinga -
pats adresas nėra tiesiogiai susiejamas su tikru asmens identifikatoriumi (pvz. asmens kodu ar paslaugos įmonės kodu).
Tokiu būdu, atributų valdymo programoje matant tik paslaugos adresą dalyvis gali nežinoti, kokią paslaugą jis autorizuoja.

Šiai problemai spręsti modelyje yra atskiras išmanusis kontraktas, kuris funkcionuoja kaip paslaugų registras. Kiekviena
paslauga, besinaudojanti modeliu, atlieka vieną transakciją į šį registrą, joje nusiųsdama savo pavadinimą bei slaptą kodą,
sugeneruotą atributų valdymo programos. Kai atributų saugyklos kontrakte bus išsaugota nauja paslaugos
P užklausa pasiekti atributą, atributų valdymo programa pagal P adresą gali kreiptis
į šį registrą ir įsitikinti, ar P tikrai užsiregistravo (rasti iš P paskyros atliktą transakciją ir įrašytą unikalų P registracijos kodą).
Taip paslaugų registro kontrakte yra saugomos tik patikimos paslaugos bei jų pavadinimai, kurie bus matomi naudotojui
atributų valdymo programoje.

Atributų valdymo programa registracijos kodą suteikia tik validžioms paslaugoms - t.y., ji turi užtikrinti,
kad tik savo tapatybę įrodžiusios paslaugos gali naudotis modeliu, taip apsisaugant nuo galimų apsimetėlių. Kaip atributų valdymo
programa tai atlieka, šiame darbe detaliai nenagrinėjama. Galimos alternatyvos pristatomos
\hypertarget{section:blockchainIDMevaluation}{\ref{section:blockchainIDMevaluation}} skyrelyje.

\subsection{Pagrindiniai naudojimo atvejai}

Šiame skyriuje UML sekų diagramomis pateikiami pagrindiniai modelio panaudos atvejai (angl. \textit{use cases}).

\subsubsection{Naudotojo adreso suteikimas paslaugai}

Kai paslauga nori pasiekti naudotojo asmens duomenis, ji paprašo naudotojo pateikti jo paskyros adresą (naudotojo patogumui, adresas rodomas
atributų valdymo programoje),
kuris yra viešas naudotojo identifikatorius blokų grandinėje.
Su gautu adresu paslauga P gali kreiptis į blokų grandinę ir 
prašyti norimų naudotojo atributų (žr.\hypertarget{fig:userGivesAddress}{~\ref{fig:userGivesAddress} pav.}).
Gavusi šį adresą, paslauga kreipiasi į atributų valdymo programą gauti šifro raktą, kurio reikės norint
dešifruoti iš blokų grandinės gautas atributų reikšmes.

Modelis neapibrėžia, kada naudotojas N pateikia adresą paslaugai P. Svarbu, kad paslauga P gautų šį adresą ir unikalų šifro raktą tarp N ir P,
nes be jų P negalės gauti ir perskaityti norimų tapatybės duomenų. Šis suteikimas galėtų būti atliekamas prisijungimo metu, įtraukiant
papildomą žingsnį.

\begin{figure}[H]
    \centering
    \includegraphics[scale=0.6]{img/userGivesAddress}
    \caption{Naudotojo adreso suteikimas paslaugai}
    \label{fig:userGivesAddress}
\end{figure}

\subsubsection{Atributo užklausa} \label{BCIDM:askForAttribute}

Paslauga P, norinti gauti tam tikrą asmens atributą (pvz. gimimo datą),
kreipiasi į atributų saugyklos išmanųjį kontraktą (žr.\hypertarget{fig:askForAttributeSequence}{~\ref{fig:askForAttributeSequence} pav.}).
Kontraktas tuomet patikrina, ar ši paslauga turi prieigą prie pageidaujamo atributo. Jei turi, tuomet grąžina šį atributą. Jis
užšifruotas paslaugos P turimu šifro raktu, tad paslauga gali jį dešifruoti ir perskaityti duomenis. Jei
naudotojas atmetęs prieigą, apie tai pranešama paslaugai. Jei naudotojas dar nesvarstęs šios prieigos, išsaugoma laukianti
(angl. \textit{pending}) paslaugos P atributo A užklausa ir laukiama naudotojo sprendimo. Naudotojas šią laukiančią užklausą galės pamatyti savo 
atributų valdymo programoje ir ten atlikti sprendimą.

\begin{figure}[H]
    \centering
    \includegraphics[scale=0.7]{img/askForAttributeSequence}
    \caption{Paslaugos atliekama atributo užklausa}
    \label{fig:askForAttributeSequence}
\end{figure}

\subsubsection{Paslaugos autorizavimas}

Atributų valdymo programa, stebinti atributų saugyklos kontraktą, praneša naudotojui apie naujas
paslaugų užklausas. Tuomet naudotojas gali pasirinkti, ar suteikti, ar atmesti prieigą prie norimo atributo
A paslaugai P. (žr.\hypertarget{fig:givePermissions}{~\ref{fig:givePermissions} pav.}). Atributų saugyklos kontraktas
asmeniui leidžia autorizuoti tik savo naudojamas paslaugas\footnote{ Ar naudotojo N1 įrašų nebando pakeisti N2, atskiriama iš
to, kas kreipiasi į funkciją.}.

Jei naudotojas sutinka suteikti prieigą, tuomet jis įveda atributo reikšmę ir atributų valdymo programa užšifruoja ją.
Užšifravus reikšmę, duomenys nusiunčiami į išmanųjį kontraktą ir jame pažymima, kad paslauga P naudotojo N autorizuota
pasiekti atributą A.

Jei naudotojas prieigą atmeta, tuomet blokų grandinėje pažymima, kad paslauga P nėra autorizuota pasiekti naudotojo N atributą A.

Jei naudotojas vėliau nuspręstų pakeisti savo sprendimą (pvz. panaikinti suteiktas prieigas prie atributo A paslaugai P),
jis šį paslaugos autorizavimo procesą galėtų pakartoti ir jau esamai prieigai.


\begin{figure}[h]
    \centering
    \includegraphics[scale=0.65]{img/givePermissions}
    \caption{Naudotojo atliekamas paslaugos autorizavimas}
    \label{fig:givePermissions}
\end{figure}

\subsubsection{Pranešimas naudotojui apie naują atributo užklausą} \label{BCIDM:blockchainMonitoring}

Norint išsaugoti naują atributo užklausą, kurią turės patvirtinti naudotojas, atributų valdymo programa turi sužinoti
apie ją iš atributų saugyklos kontrakto. Tai galima pasiekti klausantis kontrakto paskelbiamų įvykių\footnote{ Jei naudojama blokų grandinė
neturi įvykių mechanizmo, atributų valdymo programa gali periodiškai kreiptis į kontraktą ir tikrinti, ar jame yra naujų išsaugotų užklausų.}. Jeigu programa \enquote{neišgirdo}
realiu laiku paskelbto įvykio, ji gali kreiptis į kontraktą ir pasiekti anksčiau paskelbtus įvykius.

Paskelbtame užklausos įvykyje pateikiamas besikreipusios paslaugos P adresas, norimo atributo A
identifikatorius bei kurio naudotojo N atributo paslauga nori. Pagal P adresą programa kreipiasi
į paslaugų registro išmanųjį kontraktą ir gauna P registracijos duomenis (žr. \hypertarget{fig:checkForPendingPermissions}{\ref{fig:checkForPendingPermissions}} pav.).
Validavus P registraciją (P paslaugų registro
kontrakte įrašė tinkamą kodą), programa praneša naudotojui N apie naują P užklausą. Šioje užklausoje N
matys P pavadinimą ir norimą gauti atributą - iš to naudotojas galės nuspręsti, ar patvirtinti, ar atmesti šią prieigos užklausą.

Kadangi registraciją paslaugai P pakanka validuoti vieną kartą, atributų valdymo programa gali prisiminti sėkmingą P registraciją
ir tolimesnėse užklausose registracijos nebevaliduoti.

\begin{figure}[H]
    \centering
    \includegraphics[scale=0.6]{img/checkForPendingPermissions}
    \caption{Pranešimas naudotojui apie naują atributo užklausą}
    \label{fig:checkForPendingPermissions}
\end{figure}

\subsubsection{Paslaugos registracija} \label{BCIDM:serviceRegistration}

Tapatybės saugyklos išmaniajame kontrakte yra saugomi besikreipusių paslaugų adresai. Kadangi kontraktas yra viešas,
į jį kreiptis gali bet kas, turintis blokų grandinės paskyrą. Todėl, norint atskirti, kurios užklausos yra iš tikrųjų
atliktos paslaugų, modelyje yra paslaugos registracijos žingsnis (žr. \hypertarget{fig:serviceRegistration}{\ref{fig:serviceRegistration}} pav.).

Paslauga P, norinti naudotis modeliu, kreipiasi į atributų valdymo programą registracijos kodui gauti. Atributų valdymo programa
patikrina P autentiškumą ir tik įsitikinusi, kad tai nėra programišių atliekama ataka, suteikia šį kodą. Gavusi šį kodą,
P kreipiasi į paslaugų registro išmanųjį kontraktą ir į jį įrašo gautą kodą K bei P pavadinimą X. Taip registro
kontrakte būsenoje išsaugoma, kad iš P adreso įrašytas kodas K ir pavadinimas X.

Kai P atliks užklausą į atributų
saugyklos kontraktą, jame bus išsaugotas adresas, iš kurio P kreipėsi. Tuomet atributų valdymo programa pagal šį adresą
galės patikrinti, ar paslaugų registro kontrakte iš P adreso įrašytas kodas K.
Tik radus tokį įrašą, bus galima P atributo užklausą traktuoti kaip validžią ir naudoti registre įrašytą pavadinimą X.

\begin{figure}[H]
    \centering
    \includegraphics[scale=0.6]{img/serviceRegistration}
    \caption{Paslaugos registracija paslaugų registro išmaniajame kontrakte}
    \label{fig:serviceRegistration}
\end{figure}