\sectionnonum{Sąvokų apibrėžimai}
\textbf{Atributas} - charakteristika, susieta su esybe, pavyzdžiui fiziniu asmeniu. Galimi asmens atributai: gimimo data,
vardas, ūgis, pirštų antspaudai \cite{Camp2004}. Atributas gali būti laikinas (pvz. adresas) arba nuolatinis (pvz. asmens kodas).

\textbf{Identifikatorius} - tai atributas, kuris vienareikšmiškai susiejamas su jį pateikiančiu asmeniu ir kurį
sunku arba neįmanoma pakeisti. Fizinio asmens identifikatoriaus pavyzdys galėtų būti gimimo data
(žmogus gali apie ją meluoti, tačiau gimimo datos pakeisti neįmanoma) \cite{Camp2004}. Skaitmeninio identifikatoriaus
pavyzdys yra naudotojo elektroninio pašto adresas.

\textbf{Atpažinimo duomenys} (angl. \textit{credentials}) - tai duomenys, skirti asmens autentifikavimui. Jie gali būti ir asmens atributai
(pvz. biometriniai duomenys, tokie kaip pirštų antspaudai ar balsas), gali būti ir sugalvoti duomenys (pvz. slapyvardis ir slaptažodis). Dažniausiai internete naudojami
atpažinimo duomenys yra identifikatoriaus (slapyvardžio ar el. pašto) ir slaptažodžio pora \cite{Maler2008}.

\textbf{Identifikavimas} - tai procesas, kurio metu asmuo susiejamas su jo identifikatoriumi \cite{Camp2004}. Identifikavimo
pavyzdys yra asmens ir jo vardo susiejimas: \textit{tu esi Jonas Jonaitis}.

\textbf{Autentifikavimas} - tai procesas, kurio metu patvirtinama sąsaja tarp tapatybės ir jos identifikatoriaus (t.y., įrodoma,
kad asmuo iš tikrųjų yra tas, kas sakosi esąs) \cite{Camp2004, Strictest2011}. Šiam patvirtinimui naudojami atpažinimo duomenys. Autentifikavimo pavyzdys:
\textit{tavo pateiktas slapyvardis ir slaptažodis patvirtina, kad tu esi Jonas Jonaitis}.

\textbf{Autorizavimas} - tai procesas, kurio metu leidžiama arba draudžiama asmeniui atlikti konkretų veiksmą, priklausomai
nuo jo identifikatoriaus ar atributo \cite{Camp2004}. Pavyzdys: \textit{kadangi tau yra daugiau nei 18 metų, tu gali nusipirkti
energetinį gėrimą}. 

\textbf{Skaitmeninė tapatybė} - abstrakti fizinės esybės reprezentacija, sudaryta iš aibės esybės nuolatinių ar laikinų atributų,
kurie susiejami su fizine esybe \cite{Glasser2009, Camp2004}. Fizinė esybė gali būti fizinis arba juridinis asmuo.
Šiame darbe, jei nenurodyta kitaip, kalbama apie fizinio asmens skaitmeninę tapatybę.

\textbf{Skaitmeninės tapatybės valdymas} (angl. \textit{digital identity management}) - tai veiksmų, skirtų kontroliuoti
tapatybę ir su ja susijusius procesus, visuma \cite{Dabrowski2008}. Į tai įeina autentifikavimas, autorizavimas,
prieigų kontrolė, tapatybės gyvavimo ciklo
valdymas bei saugus tapatybės atributų perdavimas trečiosioms šalims \cite{Cao2010}.

\textbf{Skaitmeninių tapatybių domenas} - tai sritis, kurioje kiekviena tapatybė yra unikali. Domene esantys unikalūs
identifikatoriai leidžia sukurti vienas-su-vienu ryšį tarp šių identifikatorių ir jų tapatybių bei taip vienareikšmiškai
identifikuoti tapatybę iš jos unikalaus identifikatoriaus \cite{Josang2005}.

\textbf{Paslaugų tiekėjas} (angl. \textit{service provider}) - tai betkokia taikomoji programa, kuri suteikia naudotojui tam tikrą paslaugą ar
norimą turinį. Galimi paslaugų tiekėjai yra interneto tinklapiai, susirašinėjimo programos ar kitos taikomosios programos,
į kurias kreipiasi naudotojas \cite{Pashalidis2003, Samar1999}. Paslaugų tiekėjas gali turėti vieną ar kelias paslaugas,
kurioms reikia tapatybės valdymo funkcijų. Darbe paslaugų tiekėjas dar gali būti vadinamas pasikliaujančiąja šalimi (angl.
\textit{relying party}).

\textbf{Tapatybės tiekėjas} (angl. \textit{identity provider}) - servisas ar taikomoji programa, skirta koordinuoti su tapatybe
susijusius duomenis tarp naudotojų, jų naršyklių bei paslaugų tiekėjų \cite{Strictest2011}. Pagrindinės tapatybės tiekėjo funkcijos:
infrastruktūros naudotojų tapatybės duomenims apdoroti sukūrimas ir užklausų iš paslaugų tiekėjų bei naudotojų apdorojimas \cite{Cao2010}.

\textbf{Prieigos raktas} (angl. \textit{token}) - tai objektas, identifikuojantis skaitmeninę tapatybę \cite{TokenDefinition}.
Šis raktas būna išduodamas tapatybės tiekėjo ir skirtas identifikuoti naudotoją. Raktas
būna prisegtas prie visų autentifikuoto naudotojo užklausų ir leidžia paslaugos tiekėjui žinoti, koks naudotojas kreipiasi.

\textbf{Vienkartinis prisijungimas} (angl. \textit{single sign on}) - \textcolor{red}{to be added}.