\sectionnonum{Sąvokų apibrėžimai}
\textbf{Atributas} - charakteristika, susieta su esybe, pavyzdžiui fiziniu asmeniu. Galimi asmens atributai: gimimo data,
vardas, ūgis, pirštų antspaudai \cite{Camp2004}. Atributas gali būti laikinas (pvz. adresas) arba nuolatinis (pvz. asmens kodas).

\textbf{Identifikatorius} - tai atributas, kuris vienareikšmiškai susiejamas su jį pateikiančiu asmeniu ir kurį
sunku arba neįmanoma pakeisti. Fizinio asmens identifikatoriaus pavyzdys galėtų būti gimimo data
(žmogus gali apie ją meluoti, tačiau gimimo datos pakeisti neįmanoma) \cite{Camp2004}.

\textbf{Identifikavimas} - tai procesas, kurio metu asmuo susiejamas su jo identifikatoriumi \cite{Camp2004}. Identifikavimo
pavyzdys yra asmens ir jo vardo susiejimas: \textit{tu esi Jonas Jonaitis}.

\textbf{Autentifikavimas} - tai procesas, kurio metu patvirtinama sąsaja tarp tapatybės ir jos identifikatoriaus (t.y., įrodoma,
kad asmuo iš tikrųjų yra tas, kas sakosi esąs) \cite{Camp2004, Strictest2011}. Dažniausiai internete autentifikavimui pateikiamas identifikatorius
yra slapyvardžio ir slaptažodžio pora. Autentifikavimo pavyzdys:
\textit{tavo pateikta slapyvardžio ir slaptažodžio pora patvirtina, kad tu esi Jonas Jonaitis}.

\textbf{Autorizavimas} - tai procesas, kurio metu leidžiama arba draudžiama asmeniui atlikti konkretų veiksmą, priklausomai
nuo jo identifikatoriaus ar atributo \cite{Camp2004}. Pavyzdys: \textit{kadangi tu pateikei administratoriaus
prieigos raktą, tau leidžiama ištrinti šį tinklalapio puslapį}. 

\textbf{Skaitmeninė tapatybė} - abstrakti fizinės esybės reprezentacija, sudaryta iš aibės esybės nuolatinių ar laikinų atributų,
kurie susiejami su fizine esybe \cite{Glasser2009, Camp2004}. Fizinė esybė gali būti fizinis arba juridinis asmuo.
Šiame darbe, jei nenurodyta kitaip, kalbama apie fizinio asmens skaitmeninę tapatybę.

\textbf{Skaitmeninės tapatybės valdymas} (angl. \textit{digital identity management}) - tai procesų, skirtų kontroliuoti
tapatybę ir su ja susijusius procesus, visuma \cite{Dabrowski2008}. Į tai įeina autentifikavimas, autorizavimas,
prieigų kontrolė, tapatybės gyvavimo ciklo
valdymas bei saugus tapatybės atributų perdavimas trečiosioms šalims \cite{Cao2010}.

\textbf{Paslaugų tiekėjas} (angl. \textit{service provider}) - tai betkokia taikomoji programa, kuri suteikia naudotojui tam tikrą paslaugą ar
norimą turinį. Galimi paslaugų tiekėjai yra interneto tinklapiai, susirašinėjimo programos ar kitos taikomosios programos,
į kurias kreipiasi naudotojas \cite{Pashalidis2003, Samar1999}. Paslaugų tiekėjas gali turėti vieną ar kelias paslaugas,
kurioms reikia tapatybės valdymo funkcijų.

\textbf{Tapatybės tiekėjas} (angl. \textit{identity provider}) - servisas ar taikomoji programa, skirta koordinuoti su tapatybe
susijusius duomenis tarp naudotojų, jų naršyklių bei paslaugų tiekėjų \cite{Strictest2011}. Pagrindinės tapatybės tiekėjo funkcijos:
infrastruktūros naudotojų tapatybės duomenims apdoroti sukūrimas ir užklausų iš paslaugų tiekėjų bei naudotojų apdorojimas \cite{Cao2010}.

\textbf{Prieigos raktas} (angl. \textit{token}) - tai objektas, identifikuojantis skaitmeninę tapatybę \cite{TokenDefinition}.
Skaitmeninių tapatybių valdymo kontekste, šis raktas būna išduodamas tapatybės tiekėjo ir skirtas identifikuoti naudotoją. Raktas
būna prisegtas prie visų autentifikuoto naudotojo užklausų ir leidžia paslaugos tiekėjui žinoti, koks naudotojas kreipiasi.