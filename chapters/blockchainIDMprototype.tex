\section{Tapatybės atributų valdymo modelio prototipas}

Tapatybės atributų valdymo modelis sudarytas iš 3 dalių: atributų saugyklos kontrakto, paslaugų registro kontrakto bei
atributų valdymo programos (žr. \hypertarget{BCIDM:blockchainFunctions}{\ref{BCIDM:blockchainFunctions}} sk.).
Siekiant patvirtinti galimą modelio veikimą ir įvertinti jo kainą, naudojant \enquote{Solidity} programavimo kalbą \enquote{Ethereum} blokų grandinei buvo
realizuoti atributų saugyklos bei paslaugų registrų kontraktai. Taip pat, kadangi vienas iš naudotojų
lūkesčių skaitmeniniam tapatybės valdymui yra patogumas, jam įvertinti sukurti atributų valdymo mobiliosios programėlės maketai.

\subsection{Pasirinktos technologijos}

Išmaniesiems kontraktams programuoti pasirinkta naudoti \enquote{Ethereum} platformą ir \enquote{Solidity} programavimo
kalbą. Priežastys tam:

\begin{itemize}
    \item programuotojų bendruomenė. Nuo 2015-ųjų veikianti \enquote{Ethereum} platforma turi nemažą
    palaikančių asmenų bendruomenę, kuri prisideda prie pačio \enquote{Ethereum} atviro kodo vystymo
    bei atsako į pradedančiųjų klausimus;
    \item technologijos branda. Kai išmaniųjų kontraktų programavimas yra vis dar gana nauja sritis,
    \enquote{Ethereum} technologija, veikianti nuo 2015-ųjų, yra gana gerai dokumentuota, o rekomenduojama kontraktams rašyti programavimo kalba
    \enquote{Solidity} \cite{EthereumWhitePaper} turi keletą skirtingų kūrimo karkasų programavimo aplinkai paruošti \cite{SolidityDocumentation}.
\end{itemize}

Išmanieji kontraktai kurti naudojant \enquote{Solidity} programavimui skirtą \enquote{Truffle} kūrimo karkasą,
taikant \enquote{ganache-cli} lokalią blokų grandinę testavimui. Kontraktų funkcijos išbandytos jas kviečiant iš sukurto testinio
\enquote{React} puslapio. Kodas pasiekiamas \enquote{GitHub} repozitorijoje
\url{https://github.com/Ubaldas11/blockchain-attribute-management}.

Mobiliosios programėlės maketai sukurti naudojant \enquote{Proto.io} prototipavimo įrankį. Šis įrankis pasirinktas dėl
patogios naudotojo sąsajos, galimybės naudoti naršyklėje ir didelės paruoštų elementų kolekcijos.

\subsection{Išmanieji kontraktai}

\textbf{Atributų saugyklos kontraktas} skirtas naudotojų atributų saugojimui bei prieigų suteikimui realizuoti. Kontraktas sudarytas iš kontrakto saugyklos,
kurioje saugomos koduotos atributų reikšmės ir jų prieigos, įvykių bei funkcijų.
Kontrakto kodas pateikiamas priede \hypertarget{appendix:attributeSmartContract}{\ref{appendix:attributeSmartContract}}.

Kontrakte apibrėžti šie įvykiai:

\begin{enumerate}
    \item \textit{AccessRequested}. Skirtas pranešti, kad paslauga P prašo prieigos prie naudotojo N atributo A. Šio įvykio laukia
    atributų valdymo programa ir jį gavusi, gali pranešti naudotojui N apie naują paslaugos P užklausą.
    \item \textit{AccessChanged}. Skirtas pranešti, kad naudotojas N pakeitė paslaugos P prieigą prie atributo A. Paslaugos, 
    laukiančios šio įvykio, jį gavusios sužino apie galimai suteiktą arba panaikintą prieigą prie atributo A.
\end{enumerate}

Kontrakte apibrėžtos šios viešos funkcijos:

\begin{enumerate}
    \item \textit{grantAccess}. Skirta naudotojui N suteikti paslaugai P prieigą prie atributo A. Į funkciją paduodamas A identifikatorius,
    P adresas bei N ir P šifro raktu užkoduota A reikšmė. Funkcijoje keičiama siuntėjo atributo reikšmė - taip užtikrinama, kad naudotojas N
    negalės pakeisti naudotojo N1 atributo. Funkcija paskelbia \textit{AccessChanged} įvykį.
    \item \textit{removeAccess}. Skirta naudotojui N panaikinti paslaugos P prieigą prie atributo A. Į funkciją paduodamas A identifikatorius ir P
    adresas. Kaip ir \textit{grantAccess} atveju, keičiama siuntėjo atributo prieiga. Funkcija paskelbia \textit{AccessChanged} įvykį.
    \item \textit{requestAccess}. Skirta paslaugai P pranešti, kad ji nori prieigos prie naudotojo N atributo A reikšmės. Į funkciją paduodamas A identifikatorius
    bei N adresas. Funkcija paskelbia \textit{AccessRequested} įvykį.
    \item \textit{getAttribute}. Skirta gauti naudotojo N paslaugai P skirtą atributą A. Į funkciją paduodamas A identifikatorius, N adresas bei
    P adresas. Jei funkciją iškvietė N (t.y., pats naudotojas kreipiasi gauti suteiktą atributą A), grąžinamas siekiamas atributas. Jei kreipiasi paslauga,
    galimos 3 grąžinamos reikšmės. Jei prieiga naudotojo suteikta - grąžinama A reikšmė. Jei prieiga atmesta - pranešama, kad paslauga neautorizuota. Jei
    prieiga naudotojo dar nebuvo svarstyta, prašoma pirma kreiptis į \textit{requestAttributeAccess} funkciją.\\
    Funkcija realizuota kaip vaizdo (angl. \textit{view}) funkcija - ji nekeičia kontrakto būsenos, todėl jos kvietimas yra nemokamas.
\end{enumerate}

Kadangi įvykių paskelbimas yra pigesnis nei kontrakto būsenos keitimas, tai išnaudota paskelbiant apie naują prieigos užklausą (vietoj alternatyvos pakeisti
kontrakto būseną ir nurodyti, kad paslauga kreipėsi). Jei paslauga arba atributų valdymo programa realiu laiku \enquote{neišgirdo} įvykio, ji vėliau gali
pasiekti visą įvykių sąrašą (angl. \textit{log)}.\\

\textbf{Paslaugų registro kontraktas} skirtas užsiregistravusių paslaugų sąrašui saugoti. Paslaugų siųstuose įrašuose laikomas
registracijos kodas bei paslaugos pavadinimas.
Kontrakto kodas pateikiamas priede\hypertarget{appendix:serviceRegisterContract}{~\ref{appendix:serviceRegisterContract}}.

Kontraktą sudaro dvi funkcijos:

\begin{enumerate}
    \item \textit{register}. Skirta paslaugai užsiregistruoti. Į funkciją paduodamas registracijos kodas bei paslaugos pavadinimas, kurie išsaugomi kontrakto būsenoje.
    \item \textit{getService}. Skirta atributų valdymo programai išgauti duomenis apie besikreipusią paslaugą. Funkcijai paduodamas paslaugos adresas.
    Funkcija realizuota kaip vaizdo (angl. \textit{view}) funkcija - ji nekeičia kontrakto būsenos, todėl jos kvietimas yra nemokamas. \\ 
    Atributų valdymo programa kreipiasi į šią funkciją tuomet, kai gauta nauja paslaugos P atributo A užklausa.
    Iš gauto kodo programa gali atskirti, ar ši paslauga nėra apsimetelė
    (žr.\hypertarget{BCIDM:serviceRegister}{~\ref{BCIDM:serviceRegister} skyrelį}). Jei kodas nesutampa, atributo prieigos užklausą galima ignoruoti. Jei kodas sutampa,
    gautą pavadinimą galima rodyti atributų valdymo programoje ir pranešti apie prieigos užklausą naudotojui.
\end{enumerate}

Suprogramuotų kontraktų kainos pateikiamos \hypertarget{tab:contractPrices}{\ref{tab:contractPrices}} lentelėje. Kadangi eterio kriptovaliutos kaina
nuolat keičiasi, verta atsižvelgti, kad kainos gali kisti - čia pateikiamos kainos, matuotos 2018 gegužės 20-ą dieną. Kainos nustatytos
pasitelkus \enquote{Truffle} karkaso pateikiamą \textit{estimateGas} funkciją bei transakcijų \enquote{Ethereum} blokų grandinėje skaičiuoklę \cite{EthereumGasStation}.
Į visas eilutės tipo reikšmes paduotos 20 simbolių eilutės (priklausomai nuo kontrakte saugomų duomenų kiekio, kaina gali kisti), išskyrus
į \textit{removeAccess} funkciją, kur įrašomas \enquote{X}, pažymint panaikintą prieigą.

% \caption{Tapatybės atributų valdymo modelio išmaniųjų kontraktų kainos}
% \label{tab:contractPrices}

% Table generated by Excel2LaTeX from sheet 'SmartContractPrices'
\begin{table}[H]
    \centering
    \caption{Tapatybės atributų valdymo modelio išmaniųjų kontraktų kainos \cite{EthereumGasStation}}
      \begin{tabular}{|l|r|r|r|}
      \hline
      \multicolumn{1}{|c|}{\textbf{Vienetas}} & \multicolumn{1}{c|}{\textbf{Kuro kiekis}} & \multicolumn{1}{c|}{\textbf{Kaina, ETH}} & \multicolumn{1}{c|}{\textbf{Kaina, €}} \bigstrut\\
      \hline
      \multicolumn{1}{|p{16.5em}|}{Atributų saugyklos kontrakto sukūrimas} & 774297 & 0,0061944 & 3,72 \bigstrut[t]\\
      Funkcija \textit{getAttribute} & 0 & 0 & 0,00 \\
      Funkcija \textit{grantAccess} & 51974 & 0,0004158 & 0,25 \\
      Funkcija \textit{removeAccess} & 51543 & 0,0004123 & 0,25 \\
      Funkcija \textit{requestAttributeAccess} & 24768 & 0,0001981 & 0,12 \bigstrut[b]\\
      \hline
      \multicolumn{1}{|p{16.5em}|}{Paslaugų registro kontrakto sukūrimas} & 423418 & 0,0033873 & 2,04 \bigstrut[t]\\
      Funkcija \textit{register} & 65501 & 0,000524 & 0,31 \\
      Funkcija \textit{getService} & 0 & 0 & 0,00 \bigstrut[b]\\
      \hline
      \end{tabular}%
    \label{tab:contractPrices}%
  \end{table}%
  
  
  
  
  

\subsection{Atributų valdymo mobiliosios programėlės interaktyvus prototipas}

Siekiant patikrinti, kaip naudotojas galėtų valdyti savo asmens duomenis atributų valdymo programos pagalba, sukurtas mobiliosios programėlės naudotojo sąsajos interaktyvus prototipas. Priede \hypertarget{appendix:appSketches}{\ref{appendix:appSketches}} pateikiami pagrindiniai jo
langai. Pats interaktyvus prototipas pasiekiamas \enquote{GitHub} repozitorijoje
\url{https://github.com/Ubaldas11/blockchain-attribute-management}.

Naudotojo sąsajos prototipe asmuo nesunkiai gali apžvelgti visas suteiktas prieigas, pakeisti jas. Jau patvirtintos bei
dar laukiančios užklausos atskiriamos sekcijomis. Atributai, turintys dar nepatvirtintų užklausų žymimi su pranešimo ikonėle.
Esamą užklausų būseną galima atskirti iš naudojamų spalvų (žalia - suteikta
prieiga, raudona - atmesta, pilka - dar nesvarstyta). Nustatymų lange rodomas naudotojo paskyros adresas, kurį asmuo gali suteikti paslaugoms, norinčioms pasiekti jo atributus blokų grandinės
išmaniajame kontrakte.

Sukurtas interaktyvus prototipas leidžia teigti, kad tokia mobilioji programėlė leistų naudotojui patogiai valdyti savo
asmens duomenis.