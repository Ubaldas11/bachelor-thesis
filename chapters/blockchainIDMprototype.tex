\section{Tapatybės atributų valdymo modelio prototipas}

Tapatybės atributų valdymo modelis sudarytas iš 3 dalių (žr. \hypertarget{BCIDM:blockchainFunctions}{\ref{BCIDM:blockchainFunctions}} sk.). Siekiant įgyvendinti
aprašytą autorizavimo logiką išmaniuosiuose kontraktuose bei pamatyti galimą atributų programos vaizdą, suprogramuota autorizavimo
logikos dalis \enquote{Solidity} programavimo kalba \enquote{Ethereum} tinklui bei sukurti atributų programos maketai išmaniajai programėlei.

\subsection{Pasirinktos technologijos}

Išmaniesiems kontraktams programuoti pasirinkta naudoti \enquote{Ethereum} technologiją ir \enquote{Solidity} programavimo
kalbą išmaniesiems kontraktams rašyti. Priežastys tam:

\begin{itemize}
    \item programuotojų bendruomenė. Nuo 2015-ųjų veikianti \enquote{Ethereum} platforma turi nemažą
    palaikančių asmenų bendruomenę, kuri prisideda prie pačio \enquote{Ethereum} atviro kodo vystymo
    bei atsako į pradedančiųjų klausimus;
    \item technologijos branda. Kai išmaniųjų kontraktų programavimas yra vis dar gana nauja sritis,
    \enquote{Ethereum} technologija, veikianti nuo 2015-ųjų, yra gana gerai dokumentuota, o rekomenduojama kontraktams rašyti programavimo kalba
    \enquote{Solidity} \cite{Ethereum} turi keletą skirtingų kūrimo karkasų programavimo aplinkai paruošti \cite{SolidityDocumentation}.
\end{itemize}

Išmaniosios programėlės maketai kurti su X prototipavimo įrankiu.

\subsection{Atributų saugojimo bei autorizavimo kontraktas}

Kontraktas skirtas naudotojų atributų saugojimui bei prieigų suteikimui realizuoti. Kontraktas sudarytas iš kontrakto saugyklos,
kurioje saugomos koduotos atributų reikšmės ir jų prieigos, įvykių bei funkcijų. Kontrakto kodas pateikiamas priede \hypertarget{appendix:attributeSmartContract}{\ref{appendix:attributeSmartContract}}.

Kontraktas kurtas naudojant \enquote{Solidity} programavimui skirtą \enquote{Truffle} kūrimo karkasą,
taikant \enquote{ganache-cli} lokalią blokų grandinę testavimui. Kontrakto funkcijos išbandytos jas kviečiant iš sukurto testinio
\enquote{React} puslapio. Kodas pasiekiamas \enquote{GitHub} repozitorijoje
\url{https://github.com/Ubaldas11/blockchain-attribute-management}.

Kontrakte apibrėžti šie įvykiai:

\begin{enumerate}
    \item \textit{AccessRequested}. Skirtas pranešti, kad paslauga P prašo prieigos prie naudotojo N atributo A. Šio įvykio laukia
    atributų valdymo programa ir jį gavusi, gali pranešti naudotojui N apie naują paslaugos P užklausą.
    \item \textit{AccessChanged}. Skirtas pranešti, kad naudotojas N pakeitė paslaugos P prieigą prie atributo A. Paslaugos, 
    laukiančios šio įvykio, jį gavusios sužino apie galimai suteiktą arba panaikintą prieigą prie atributo A.
\end{enumerate}

Kontrakte apibrėžtos šios viešos funkcijos:

\begin{enumerate}
    \item \textit{grantAccess}. Skirta naudotojui N suteikti paslaugai P prieigą prie atributo A. Į funkciją paduodamas A identifikatorius,
    P adresas bei N ir P šifro raktu užkoduota A reikšmė. Funkcijoje keičiama siuntėjo atributo reikšmė - taip užtikrinama, kad naudotojas N
    negalės pakeisti naudotojo N1 atributo. Funkcija paskelbia \textit{AccessChanged} įvykį;
    \item \textit{removeAccess}. Skirta naudotojui N panaikinti paslaugos P prieigą prie atributo A. Į funkciją paduodamas A identifikatorius ir P
    adresas. Kaip ir \textit{grantAccess} atveju, keičiama siuntėjo atributo prieiga. Funkcija paskelbia \textit{AccessChanged} įvykį;
    \item \textit{requestAccess}. Skirta paslaugai P pranešti, kad ji nori prieigos prie naudotojo N atributo A reikšmės. Į funkciją paduodamas A identifikatorius
    bei N adresas. Funkcija paskelbia \textit{AccessRequested} įvykį;
    \item \textit{getAttribute}. Skirta gauti naudotojo N paslaugai P skirtą atributą A. Į funkciją paduodamas A identifikatorius, N adresas bei
    P adresas. Jei funkciją iškvietė N (t.y., pats naudotojas kreipiasi gauti suteiktą atributą A), grąžinamas siekiamas atributas. Jei kreipiasi paslauga,
    galimos 3 grąžinamos reikšmės. Jei prieiga naudotojo suteikta - grąžinama A reikšmė. Jei prieiga atmesta - pranešama, kad paslauga neautorizuota. Jei
    prieiga naudotojo dar nebuvo svarstyta, prašoma pirma kreiptis į \textit{requestAttributeAccess} funkciją.\\
    Funkcija realizuota kaip vaizdo (angl. \textit{view}) funkcija - ji nekeičia kontrakto būsenos, todėl jos kvietimas yra nemokamas.
\end{enumerate}

Kadangi įvykių paskelbimas yra pigesnis nei kontrakto būsenos keitimas, tai išnaudota paskelbiant apie naują prieigos užklausą (vietoj alternatyvos pakeisti
kontrakto būseną ir nurodyti, kad paslauga kreipėsi). Jei paslauga arba atributų valdymo programa realiu laiku \enquote{neišgirdo} įvykio, ji gali
pasiekti visą įvykių sąrašą (angl. \textit{log)}.

\subsection{Atributų valdymo programos maketai}

Šauni programėlė.