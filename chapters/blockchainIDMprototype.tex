\section{Tapatybės atributų valdymo modelio prototipas}

Tapatybės atributų valdymo modelis sudarytas iš 3 dalių (žr. \hypertarget{BCIDM:blockchainFunctions}{\ref{BCIDM:blockchainFunctions}} sk.). Siekiant įgyvendinti
aprašytą autorizavimo logiką išmaniuosiuose kontraktuose bei pamatyti galimą atributų programos vaizdą, suprogramuota autorizavimo
logikos dalis \enquote{Solidity} programavimo kalba \enquote{Ethereum} tinklui bei sukurti atributų programos maketai išmaniajai programėlei.

\subsection{Pasirinktos technologijos}

Išmaniesiems kontraktams programuoti pasirinkta naudoti \enquote{Ethereum} technologiją ir \enquote{Solidity} programavimo
kalbą išmaniesiems kontraktams rašyti. Priežastys tam:

\begin{itemize}
    \item programuotojų bendruomenė. Nuo 2015-ųjų veikianti \enquote{Ethereum} platforma turi nemažą
    palaikančių asmenų bendruomenę, kuri prisideda prie pačio \enquote{Ethereum} atviro kodo vystymo
    bei atsako į pradedančiųjų klausimus;
    \item technologijos branda. Kai išmaniųjų kontraktų programavimas yra vis dar gana nauja sritis,
    \enquote{Ethereum} technologija, veikianti nuo 2015-ųjų, yra gana gerai dokumentuota, o rekomenduojama kontraktams rašyti programavimo kalba
    \enquote{Solidity} \cite{Ethereum} turi keletą skirtingų kūrimo karkasų programavimo aplinkai paruošti \cite{SolidityDocumentation}.
\end{itemize}

Išmaniosios programėlės maketai kurti su X prototipavimo įrankiu.

\subsection{Išmaniųjų kontraktų funkcijos}

Šaunios funkcijos.

\subsection{Atributų valdymo programos maketai}

Šauni programėlė.