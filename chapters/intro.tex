\sectionnonum{Įvadas}
Interneto paslaugos šiais laikais yra neatsiejama žmonių gyvenimo dalis.
Norėdami individualizuoti turinį, sustiprinti taikomosios programos saugumą ar siekdami
iš anksto išvengti kenkėjiškų tikslų turinčių asmenų ar sukurtų robotų, paslaugų tiekėjai
siekia identifikuoti savo naudotojus. Interneto naudotojų skaičiui perkopus 4 milijardus \cite{InternetUsers2018},
o kiekvienam naudotojui vidutiniškai turint po 7 skirtingas socialines paskyras \cite{Mander2017}, asmenų
tapatybių valdymas, autentifikavimas ir autorizavimas tampa vis didesniu iššūkiu.

Tapatybių valdymas kelia problemų naudotojams. Bene didžiausi atsiradę keblumai:
milžiniškas įsimintinų slaptažodžių kiekis bei sunkumai kontroliuojant savo asmens duomenų sklaidą
skirtingose sistemose. Vidutiniškai interneto naudotojas turi 25 slaptažodžių reikalaujančias paskyras
ir per dieną turi įvesti 8-is slaptažodžius \cite{Florencio2007}. Susidarius tokiai situacijai, per didelis įsimintinų slaptažodžių kiekis neretai 
priverčia naudotojus paaukoti saugumą dėl patogumo
ir pradėti naudoti tą patį slaptažodį sirtingoms sistemoms \cite{Pashalidis2003, Samar1999}. Naudotojas, turėdamas keletą
paskyrų skirtingose sistemose, taip pat praranda dalį savo asmens duomenų kontrolės. Jam tenka pasitikėti
taikomosios programos naudojamomis technologijomis ir metodais bei tikėtis, kad jie bus pakankamai saugūs
ir stabilūs, o suteikti asmens duomenys nepasieks nepageidaujamų adresatų. Didėjant naudojamų paslaugų kiekiui,
naudotojo skaitmeninės tapatybės duomenis turi vis daugiau taikomųjų programų ir bent vienai iš jų
patyrus programišių įsilaužimą ar kitokią nesėkmę, jautrūs naudotojo duomenys gali būti paviešinti. 

Taikomi skirtingi metodai skaitmeninei tapatybei internete valdyti. 
Šiais laikais vienas dažniausiai internete sutinkamų sprendimų yra jungtinis (angl. \textit{federated}) tapatybės valdymas ir jo suteikiamas vienkartinis prisijungimas (angl. \textit{single sign-on}).
Šis sprendimas leidžia naudotojui pasirinkti
vieną tapatybės tiekėją (angl. \textit{identity provider}) ir patikėti jam skaitmeninės tapatybės valdymą. Tokiu
būdu naudotojui pakanka turėti paskyrą tik tapatybės tiekėjo sistemoje, o kreipiantis į paslaugas naudotis jau turima tapatybės tiekėjo paskyra.
Tačiau šis sprendimo būdas taip pat turi aiškių trūkumų: naudotojas negali prisijungti
prie paslaugų, nepalaikančių pasirinkto tapatybės tiekėjo, jo pasiekiamumas
tampa vieninteliu nesėkmės tašku (angl. \textit{single point of failure}), naudotojas taip pat praranda dalį savo asmens duomenų kontrolės.
Naršantis internete asmuo yra priverstas pasitikėti tapatybės tiekėjo
gebėjimu perduoti tik naudotojo leistus asmens duomenis ir tik toms trečiosioms šalims, kurias jis patvirtina.
Kaip rodo \textit{Cambridge Analytica} incidentas \cite{CambridgeAnalytica}, net didžiosios kompanijos, tokios
kaip \enquote{Facebook}, ne visada sugeba tai užtikrinti.

Blokų grandinė (angl. \textit{blockchain}) yra nauja alternatyva skaitmeninės tapatybės valdymui. Ši technologija veikia kaip
paskirstytų įrašų platforma (angl. \textit{distributed ledger platform}), kurioje kiekvienas įrašas yra nekintamas, o visi
užfiksuoti įrašai atspindi tikslią transakcijų istoriją nuo pat grandinės sukūrimo \cite{Baars2016}. Saugant tapatybės duomenis šioje grandinėje ir
pritaikius reikiamą blokų grandinės pasiekiamumo lygį įrašų rašymui ir skaitymui, asmuo visada
žinotų, kokia trečioji šalis gali pasiekti kokius tapatybės duomenis. Kadangi blokų grandinė yra decentralizuota, pritaikius ją skaitmeninės tapatybės valdyme taip pat
būtų galima išvengti šioje srityje dažnos vienintelio nesėkmės taško problemos. 

Šiame darbe blokų grandinės tinkamumas skaitmeninės tapatybės valdymui tiriamas iš naudotojo perspektyvos.
Pateikus esminius naudotojų poreikius identiteto valdymui,
apžvelgiamas dabar naudojamų sistemų gebėjimas įgyvendinti šiuos reikalavimus. Apibendrinus pagrindines neišspręstas naudotojams kylančias problemas,
tiriama, kaip blokų grandinė gali padėti jas įveikti, kokie šios technologijos panaudojimo skaitmeniniame tapatybės valdyme
pranašumai, trūkumai bei pritaikymo barjerai (angl. \textit{adoption barriers}).
\\

\textbf{Darbo tikslas} - įvertinti blokų grandinės tinkamumą skaitmeninės tapatybės valdymui.
\\

\textbf{Darbe keliami uždaviniai}:

\begin{enumerate}
    \item Apžvelgti skaitmeninės tapatybės valdymo sprendimus, vertinant jų gebėjimą įgyvendinti naudotojų poreikius.
    \item Įvardyti naudotojams kylančias problemas skaitmeninės tapatybės valdyme.
    \item Apžvelgti blokų grandinės technologiją ir jos savybes.
    \item Atsižvelgus į blokų grandinės savybes ir esamą jos taikymą tapatybės srityje.
    įvertinti blokų grandinės gebėjimą spręsti įvardytas naudotojų problemas.
    \item Pateikti blokų grandinės panaudojimo atvejį skaitmeninės tapatybės valdymui ir sukurti jo veikimą demonstruojantį prototipą.
    \item Įvertinti pateiktą sprendimą apibūdinant jo privalumus, trūkumus ir galimus pritaikymo barjerus.
\end{enumerate}