\sectionnonum{Įvadas}
Naudojimasis interneto paslaugomis šiais laikais yra neatsiejama žmonių gyvenimo dalis.
Norėdami individualizuoti turinį, sustiprinti taikomosios programos saugumą ar siekdami
iš anksto išvengti kenkėjiškų tikslų turinčių asmenų ar sukurtų robotų, paslaugų tiekėjai
neretai prašo naudotojų autentifikuotis. Interneto naudotojų skaičiui perkopus 4 milijardus \cite{InternetUsers2018},
o kiekvienam naudotojui vidutiniškai turint po 7 skirtingas socialines paskyras \cite{Mander2017}, asmenų
autentifikavimas tampa vis didesniu iššūkiu.

Susidariusi situacija kelia problemų tiek paslaugų tiekėjams, tiek jų naudotojams. Kiekvienas
paslaugų tiekėjas turi skirti papildomų resursų naudotojų tapatybių valdymui, jų autentifikavimui,
jautrių duomenų saugumo užtikrinimui. Paslaugų naudotojams bene didžiausi atsiradę keblumai:
milžiniškas įsimintinų slaptažodžių kiekis bei sunkumai kontroliuojant savo asmens duomenų sklaidą
skirtingose sistemose. Vidutiniškai interneto naudotojas turi 25 paskyras, reikalaujančias slaptažodžių
ir per dieną turi įvesti 8-is slaptažodžius \cite{Florencio2007}. Susidarius tokiai situacijai, per didelis įsimintinų slaptažodžių kiekis neretai 
priverčia naudotojus paaukoti saugumą dėl patogumo
ir pradėti naudoti tą patį slaptažodį sirtingoms sistemoms \cite{Pashalidis2003, Samar1999}. Naudotojas, turėdamas keletą
paskyrų skirtingose sistemose, taip pat praranda dalį savo asmens duomenų kontrolės. Jam tenka pasitikėti
taikomosios programos naudojamomis technologijomis ir metodais ir tikėtis, kad jie bus pakankamai saugūs
ir stabilūs bei suteikti asmens duomenys nepasieks nepageidaujamų adresatų. Didėjant naudojamų paslaugų kiekiui,
naudotojo skaitmeninės tapatybės duomenis turi vis daugiau taikomųjų programų ir bent vienai iš jų
patyrus programišių įsilaužimą ar kitokią nesėkmę, jautrūs naudotojo duomenys būna paviešinti. 

Vienu iš pagrindinių skaitmeninės tapatybės valdymo keliamų problemų sprendimu išlieka vienkartinis prisijungimas
 (angl. \textit{Single Sign-On}). Šis sprendimas leidžia naudotojui pasirinkti
vieną tapatybės tiekėją (angl. \textit{identity provider}) ir patikėti jam skaitmeninės tapatybės valdymą. Tuomet naudotojas prie visų
paslaugų, palaikančių pasirinkto tapatybės tiekėjo (pvz. \textit{Facebook}) prisijungimą, gali autentifikuotis naudodamas ta pačia paskyra. Tokiu
būdu naudotojui pakanka prisiminti tik slaptažodžius, užregistruotus tapatybės tiekėjų sistemose, o paslaugų
tiekėjai neturi patys rūpintis autentifikavimu ar autorizavimu, o jį užtikrina integruodami sistemą
su tapatybės tiekėju. Tačiau šis sprendimo būdas taip pat turi aiškių trūkumų: naudotojas negali prisijungti
prie paslaugų, nepalaikančių pasirinkto tapatybės tiekėjo, dėl paslaugų tiekėjų priklausomybės nuo tapatybės tiekėjo pastarojo pasiekiamumas
tampa vieninteliu nesėkmės tašku (angl. \textit{single point of failure}), naudotojas taip pat praranda dalį savo asmens duomenų kontrolės.
Naršantis internete asmuo yra priverstas pasitikėti tapatybės tiekėjo
gebėjimu perduoti tik naudotojo leistus asmens duomenis ir tik toms trečiosioms šalims, kurias jis patvirtina.
Kaip rodo \textit{Cambridge Analytica} neleistino duomenų perdavimo incidentas \cite{CambridgeAnalytica}, net didžiosios kompanijos, tokios
kaip \textit{Facebook}, ne visada sugeba tai užtikrinti.

Blokų grandinė (angl. \textit{blockchain}) yra nauja alternatyva skaitmeninės tapatybės valdymui. Ši technologija veikia kaip
paskirstytų įrašų platforma (angl. \textit{Distributed Ledger Platform}), kurioje kiekvienas įrašas yra nekintamas (angl. \textit{immutable}), o visi
užfiksuoti įrašai atspindi tikslią transakcijų istoriją nuo pat grandinės sukūrimo \cite{Baars2016}. Saugant tapatybės duomenis šioje grandinėje ir
pritaikius reikiamą blokų grandinės pasiekiamumo lygį įrašų rašymui ir skaitymui, asmuo visada
žino, kokia trečioji šalis gali pasiekti kokius tapatybės duomenis. Kadangi blokų grandinė yra decentralizuota, pritaikius ją skaitmeninių tapatybių valdyme taip pat
būtų galima išvengti šioje srityje dažnos vienintelio nesėkmės taško problemos. Šiame darbe nagrinėjama, kada verta naudoti blokų grandinę
naudotojų skaitmeniniam autentifikavimui bei autorizavimui, kokie to pranašumai, trūkumai bei priėmimo barjerai (angl. \textit{adoption barriers}).
\\

\textbf{Darbo tikslas} - išanalizuoti blokų grandinės tinkamumą skaitmeninės tapatybės valdymui.
\\

\textbf{Darbe keliami uždaviniai}:

\begin{enumerate}
    \item Išnagrinėti esamus skaitmeninės tapatybės valdymo sprendimus ir jų keliamus iššūkius.
    \item Apibūdinti blokų grandines ir jų savybes, leidžiančias spręsti dabartines identifikavimo problemas.
    \item \textit{Apžvelgti esamas blokų grandinės sistemas, taikomas naudotojų autentifikavimui ar autorizavimui}.
    \item Pateikti blokų grandinės panaudos atvejį skaitmeninės tapatybės valdymui ir sukurti jo veikimą demonstruojantį prototipą.
    \item Įvertinti pateikto sprendimo tinkamumą apibūdinant jo privalumus, trūkumus ir pritaikymo barjerus.
    \item Palyginti pristatytą sprendimą su standartiniais naudotojų autentifikavimo ir autorizavimo būdais.
\end{enumerate}

\textbf{Laukiami rezultatai}:

\begin{enumerate}
    \item Identifikuoti pagrindiniai dabar kylantys skaitmeninės tapatybės valdymo iššūkiai.
    \item Nustatyta, kuomet blokų grandinės technologija yra tinkama skaitmeninio identiteto problemoms spręsti.
    \item Sukurtas tapatybės valdymo sistemos prototipas, naudojantis blokų grandinę.
    \item Įvertinti pateiktos sistemos pranašumai bei trūkumai.
\end{enumerate}