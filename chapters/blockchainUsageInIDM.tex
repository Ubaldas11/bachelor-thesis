\section{Blokų grandinės naudojimas skaitmeninės tapatybės valdyme} \label{section:BCusageinIDM}

Šiame skyriuje nagrinėjamas blokų grandinės technologijos panaudojimas tapatybės valdyme.
Tiriama, kaip blokų grandinės savybės gali padėti išspręsti egzistuojančias
naudotojų problemas, apžvelgiami technologiją identiteto valdyme jau taikantys projektai.


\subsection{Blokų grandinės savybių tinkamumas identiteto valdymui} \label{blockchain:suitabilityForIDM}

\hypertarget{section:blockchainOverview}{\ref{section:blockchainOverview}} skyriuje apibendrintos naudotojų problemos
tapatybės valdyme rodo, kad naudotojams internete trūksta savo asmens duomenų kontrolės, jie turi pasitikėti paslaugų programų veikimu
bei tapatybės tiekėjo pasiekiamumu. Viešos blokų grandinės savybės atveria įvairių galimybių, kaip spręsti susidariusias problemas:

\begin{enumerate}
    \item Decentralizuota sistema eliminuoja vienintelio nesekmės taško situaciją. Pritaikius decentralizuotą blokų grandinę
    tapatybės valdymo sistemoje, dėl blokų grandinės išskirstymo po skirtingus mazgus 
    tapatybės tiekėjas nebebūtų vienintelis nesekmės taškas identiteto valdymo infrastruktūroje. Tokiu būdu tapatybės informacija būtų prieinama ir tada,
    kai tapatybės tiekėjas yra nepasiekiamas.

    \item Visiems prieinami grandinės įrašai ir sistemos veikimas padidina skaidrumą. Blokų grandinėje saugant naudotojo suteiktas prieigas skirtingoms paslaugoms,
    naudotojas betkada galėtų matyti paslaugoms suteiktas teises. Prieigų suteikimą aprašius išmaniuosiuose kontraktuose, paslaugos duomenis
    pasiekti galėtų tik tada, kai naudotojas išreikštinai tai patvirtina.

    \item Nekintamumas apsaugo nuo galimų įsilaužimų siekiant pakeisti įrašus.
    Blokų grandinėje saugant naudotojo atributus ar paslaugų prieigos teises,
    o rašymo teises į grandinę turint tik naudotojui ar jo pasirinktoms programoms, naudotojas būtų užtikrintas, kad
    neįvyko jo paties neautorizuotų pakeitimų.
\end{enumerate}

Įvardyti blokų grandinės pranašumai būtų sunkiau įgyvendinami taikant įprastą centralizuotą duomenų bazę. Tokia duomenų bazė
būtų kontroliuojama vienos organizacijos ar asmens, tad išliktų vienintelis nesekmės taškas. Įrašų prieinamumu ir nekintamumu bei veikimo
skaidrumu
turėtų išreikštinai rūpintis duombazės kūrėjai, kai šias savybės užtikrina pati blokų grandinės technologija. Pranašumas, kurį
šiuo atveju turi duomenų bazė, yra plečiamumas - saugoti didžiulius kiekius tapatybės duomenų blokų grandinėje yra gana brangu.

\hypertarget{blockchain:concerns}{~\ref{blockchain:concerns}} skyrelyje įvardyti pagrindiniai blokų grandinės apribojimai nesudaro didelių kliūčių šią technologiją taikyti skaitmeninės tapatybės valdyme.
Anonimiškumo klausimą galima spręsti šifruojant saugomus tapatybės duomenis,
o raktus dešifravimui suteikiant tik norimiems asmenims. Blokų grandinės greitaveika neturėti kelti problemų
standartinėms tapatybės valdymo sistemos operacijoms \cite{Lo2017}, o didėjant įrašų kiekiui dalį jų būtų galima iškelti į
šalutinę grandinę (angl. \textit{off-chain}). Daugumos ataka decentralizuotoje viešoje blokų grandinėje išlieka kaip galimas pavojus,
tačiau šią riziką galima sumažinti pasirenkant tokią blokų grandinę, kurioje tokios atakos vykdymo kaina būtų per didelė.

\subsection{Blokų grandinės projektai tapatybės valdymo srityje} \label{section:relatedWork}

Tiriant blokų grandinės naudojimą skaitmeninės tapatybės valdymo srityje, rasta įvairių
technologiją bandančių pritaikyti projektų. Šis skyrius skirtas trumpai apžvelgti autoriaus nuomone
įdomiausius projektus, susijusius su tapatybe bei blokų grandinės technologija.

\subsubsection{Blockstack}

\enquote{Blockstack} projektas yra kompiuterių tinklas, kolektyviai saugantis globalų
domenų vardų bei jų viešų raktų registrą. Su šiuo registru, \enquote{Blockstack} įgyvendiną
decentralizuotą domenų vardų sistemą, kur kiekvienas norintis gali užregistruoti savo domeną \cite{BlockstackWhitepaper}.
\enquote{OneName} projektas, naudojantis \enquote{Blockstack} sistemą, siekia sukurti decentralizuotą tapatybę,
kurią susikūręs naudotojas galėtų atsisakyti visų kitų turimų socialinių tinklų (\enquote{Facebook},
\enquote{Google}) tapatybių.

\subsubsection{UPort}

\enquote{Consensys} įmonės \enquote{UPort} projektas panašus į \enquote{OneName} - jis skirtas naudotojams turėti decentralizuotas
skaitmenines tapatybes, registruotas \enquote{Ethereum} tinkle, su kuriomis galėtų atlikti \enquote{Ethereum} transakcijas.
Naudotojai su \enquote{UPort} paskyra
galėtų prisijungti prie paslaugų bei valdyti savo atributus. Šio darbo rašymo metu \enquote{UPort} turi
parengę integracijas paslaugoms ir elektroninę piniginę, skirtą \enquote{Ethereum} raktų
ir transakcijų valdymui, tačiau mobilioji programėlė \enquote{UPort} skaitmeninei tapatybei valdyti vis dar ruošiama (angl.
\textit{under development}) \cite{UPort}.

\subsubsection{Tradle}

\enquote{Tradle} yra blokų grandine paremtas projektas, skirtas finansinėms institucijoms (bankams,
draudimo įmonėms) įgyvendinti \enquote{pažink savo klientą} (angl. \textit{Know Your Customer - KYC}) procesus,
suteikiant klientams didesnę asmens duomenų kontrolę. \enquote{Tradle} suteikia
klientams programinę įrangą, kuri padės integruoti \enquote{Tradle} platformą su jų įmonėse jau egzistuojančia klientų duomenų saugojimo
infrastruktūra. Klientams išreikštinai sutikus, \enquote{Tradle} taip pat suteikia galimybę perduoti asmenų duomenis iš vienos
finansinės įmonės į kitą \cite{Baars2016}.

\subsection{Apibendrinimas}

Blokų grandinės savybės suteikia pagrindo naudoti šią technologiją skaitmeninės tapatybės valdyme. Decentralizuotumas, skaidrumas
bei nekintamumas gali padėti grąžinti naudotojams jų pačių asmens duomenų kontrolę bei panaikinti didžiulę
priklausomybę nuo tapatybės ar paslaugų tiekėjų. Jau egzistuojantys projektai, taikantys blokų grandinę identiteto valdyme,
įrodo technologijos galimą potencialą šioje srityje.